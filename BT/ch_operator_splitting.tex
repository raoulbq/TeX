\chapter{Time propagation by operator splitting}
\label{ch_operator_splitting}

In this chapter we will present an algorithm based on an analytical approximation
of the matrix exponential in the time propagation operator.

The analytic explicit time propagation for an initial value of the time-dependent
Schrödinger equation \eqref{eq:basics_tdse_semi} can be written as

\begin{equation} \label{eq:analytic_time_propagation}
  \Ket{\psi\ofs{x,t}} = \exp\ofs{-\frac{i}{\varepsilon^2}H t} \Ket{\psi\ofs{x,0}}
\end{equation}

In our case the Hamiltonian $H$ is a square matrix of small size. Let's write this
expression in a more detailed fashion

\begin{equation}
  \exp\ofs{-\frac{i}{\varepsilon^2}H t}
  = \exp\ofs{-\frac{i}{\varepsilon^2}\left(T+V\right) t}
  = \exp\ofs{\frac{-i}{\varepsilon^2}T t + \frac{-i}{\varepsilon^2}V t}
\end{equation}

where the kinetic and potential operators $T$ and $V$ are given by \eqref{eq:basics_def_ops}.


\section{Operator splitting}

Given a linear ordinary differential equation

\begin{equation}
\begin{split}
  \dot{u} & = \left(A + B\right) u \\
 u\ofs{0} & = u_0 \\
\end{split}
\end{equation}

with explicit solution

\begin{equation}
  u\ofs{t} = e^{\left(A+B\right)t} \cdot u_0
\end{equation}

If $A$ and $B$ were matrices it's not possible to calculate the exponential in general.
Only if $A$ and $B$ commute we can apply the Baker-Campbell-Hausdorff formula
and write $e^{\left(A+B\right)}$ as $e^{A}e^{B}$. However in our application the two
operators $T$ and $V$ do never commute.

A possible approximation is an operator splitting. See reference \cite{J_C_operator_splitting}
and \cite{Gradinaru_strangsplitting} for the details of this ansatz. A time-stepping
scheme for small but finite time steps $\tau$ is:

\begin{equation} \label{eq:general_operator_splitting}
  u\ofs{t+\tau} \approx e^{\frac{1}{2}\tau B} e^{\tau A} e^{\frac{1}{2}\tau B} u\ofs{t} \,.
\end{equation}

This is called a symmetric Lie-Trotter splitting. Now we apply this scheme to the
equation \eqref{eq:analytic_time_propagation} with $A \assign -\frac{i}{\varepsilon^2} T$
and $B \assign -\frac{i}{\varepsilon^2} V$. This yields for the propagation operator

\begin{equation}
  e^{\frac{1}{2} \tau \frac{-i}{\varepsilon^2} V}
  e^{\tau \frac{-i}{\varepsilon^2} T}
  e^{\frac{1}{2} \tau \frac{-i}{\varepsilon^2} V}
  + \bigO{\tau^3}
\end{equation}

which is of locally third order. The approximative time propagation of the Schrödinger
equation reads then

\begin{equation} \label{eq:operator:splitting_propagation}
  \Ket{\psi\ofs{t + \tau}}
  =
  e^{-\frac{i}{2\varepsilon^2} \tau V}
  e^{-\frac{i}{\varepsilon^2} \tau T}
  e^{-\frac{i}{2\varepsilon^2} \tau V}
  \Ket{\psi\ofs{t}}
\end{equation}


\section{The propagation algorithm}

With the operator splitting we gained the chance to perform each of these three propagation
steps individually. This comes in handy, we can utilize the fact that the kinetic
operator $T$ is diagonal in momentum space while the potential $V$ is diagonal in
position space. Thus we switch to momentum space for the propagation by $e^{-\frac{i}{\varepsilon^2} \tau T}$
and back to position space afterwards. This change of basis is done by a Fourier
transformation and can be performed efficiently by a fast Fourier transform
algorithm. With all these parts put together the time propagation now reads

\begin{equation} \label{eq:Fourier_propagation_simple}
  \Ket{\psi\ofs{x,t+\tau}}
  =
  e^{-\frac{i}{2\varepsilon^2} \tau V\ofs{x}}
  \ift{
  e^{-\frac{i}{\varepsilon^2} \tau T\ofs{\omega}}
  \ft{
  e^{-\frac{i}{2\varepsilon^2} \tau V\ofs{x}}
  \Ket{\psi\ofs{x,t}}}}
\end{equation}

where $\ft{\cdot}$ denotes a formal Fourier transformation. This formula describes
how to advance a single time step of duration $\tau$. The whole algorithm consists
of as many iterations of this formula as desired.

A further simplification of \eqref{eq:Fourier_propagation_simple} is possible when
we take into account the exact expression for $T$ given by equation \eqref{eq:basics_def_ops}
with the mass $m$ set to $1$. The result is $e^{-\frac{i}{\varepsilon^2} \tau T\ofs{\omega}}
= e^{-i\varepsilon^2 \tau \widetilde{T}\ofs{\omega}}$ with $\widetilde{T} \assign -\frac{1}{2}\frac{\partial^2}{\partial x^2}$.

Both the kinetic as well as the potential operator are time-independent and we can
precalculate their exponentials. This reduces the per iteration cost of the time-
stepping algorithm by a big amount.

The expression $-\frac{i}{2\varepsilon^2} \tau V\ofs{x}$ easily evaluates
to a scalar for any given $x$. This won't cause any troubles in the exponential.
For the kinetic operator we need to go to Fourier space. Using the linearity of $\mathcal{F}$
and the above definition for $\widetilde{T}$:

\begin{equation*}
  \ft{-i \varepsilon^2 \tau \widetilde{T}}
  = -i \varepsilon^2 \tau \ft{\widetilde{T}}
  = -i \varepsilon^2 \tau \ft{-\frac{1}{2}\frac{\partial^2}{\partial x^2}}
\end{equation*}

As known from Fourier theory it holds that $\ft{f^{\left(n\right)}\ofs{x}} = \left(i\omega\right)^n \ft{f\ofs{x}}$
for the $n$\textsuperscript{th} derivatives of a suitable function $f$ and with $\omega$ being the
Fourier variable. Use of this identity gives

\begin{equation*}
\begin{split}
  \ft{-\frac{1}{2}\frac{\partial^2}{\partial x^2} f\ofs{x}}
%   & = -\frac{1}{2} \ft{\frac{\partial^2}{\partial x^2} f\ofs{x}} \\
  & = -\frac{1}{2} \left(i\omega\right)^2 \ft{f\ofs{x}}
\end{split}
\end{equation*}

If we now simplify this result further, we get $\ft{\widetilde{T}} = -\frac{1}{2} \left(i\omega\right)^2$
and thus

\begin{equation}
\begin{split}
  \ft{-i \varepsilon^2 \tau \widetilde{T}}
%   & = -i \varepsilon^2 \tau \left(-\frac{1}{2}\right) \left(i\omega\right)^2 \\
  & = -\frac{i}{2} \varepsilon^2 \tau \omega^2
\end{split}
\end{equation}

We got rid of the partial derivatives in the exponent at the cost of an additional
real scalar $\omega$. Finally we introduce the following notation

\begin{equation} \label{eq:def_Fourier_operator_exponentials}
\begin{split}
  T_e\ofs{\omega} & \assign \exp\ofs{-\frac{1}{2}i\varepsilon^2\tau\omega^2} \\
  V_e\ofs{x}      & \assign \exp\ofs{-\frac{i}{2\varepsilon^2}\tau V\ofs{x}}
\end{split}
\end{equation}

for the precalculated propagation operators. Both exponentials only contain numbers.
With these definitions the algorithm given by equation \eqref{eq:Fourier_propagation_simple} becomes

\begin{equation} \label{eq:Fourier_propagation_alg}
  \psi_{n+1}\ofs{x} =
  V_e\ofs{x} \mathcal{F}^{-1} \Bigl(
    T_e\ofs{\omega} \underbrace{ \mathcal{F}\bigl(
      V_e\ofs{x} \cdot \psi_n\ofs{x}
    \bigr)}_{\widehat{\psi}_n\ofs{\omega}}
  \Bigr)
\end{equation}

where $\psi_n$ is a short notation for $\psi\ofs{x, t = n\tau}$ with $n$
denoting the number of the current time step.


\subsection{The discretized space}

For the numerical simulation we first define the computational
domain $\Omega \in \mathbb{R}^d$. In our case where we deal with only one space
dimension and $d=1$, this domain is simply an interval on the real line. Without
loss of generality we can assume $\Omega$ to be centred around the origin and bounded
by $\alpha$. To simplify further, also with respect to the Fourier transformation,
we write the constant $\alpha$ as a multiple of $\pi$ and denote the scaling factor
by $f \in \mathbb{R}$.

\begin{equation} \label{eq:comp_domain}
  \Omega \assign \left[ -f \pi, f \pi \right]
\end{equation}

The domain is still part of the continuum. For the numerical computation we need
to discretize the space and introduce a grid on $\Omega$. Denote the number of grid
nodes by $n$ and the grid spacing by $h$. Then a grid $\Gamma$ is given by

\begin{gather} \label{eq:comp_domain_grid}
  \Gamma \assign \{ \gamma_0 < \gamma_1 < \ldots < \gamma_{N-1} \} \\
  \gamma_0 \equiv -f \pi \quad \text{and} \quad \gamma_{N-1} \equiv f \pi
\end{gather}

with equidistant grid nodes $\gamma_i \in \Omega$. Additionally to the grid in position space
we need a grid in the Fourier space for computing $T_e\ofs{\omega}$. Suppose the
number $n$ of grid nodes is given as a power of $2$, then we have $n = 2^{k}$. Now
the grid $\widehat{\Gamma}$ in Fourier space is given as

\begin{gather} \label{eq:comp_domain_grid_fourier}
  \widehat{\Gamma} \assign \{ \omega_0 < \omega_1 < \ldots < \omega_{N-1} \} \\
  \omega_0 \assign 1-2^{k-1} \quad \text{and} \quad \omega_{N-1} \assign 2^{k-1}
\end{gather}

Depending on the implementation of the discrete Fourier transform we want to use
we have to shift the nodes with negative sign and reorder the set $\widehat{\Gamma}$.

\subsection{Discretized time evolution operators}

The time propagation operator exponentials $T_e$ and $V_e$ given by equation \eqref{eq:def_Fourier_operator_exponentials}
become

\begin{equation}
\begin{split}
  \widetilde{T}_e\ofs{\widetilde{\omega}} & \assign \exp\ofs{-\frac{1}{2}i\varepsilon^2\tau\widetilde{\omega}^2} \\
  \widetilde{V}_e\ofs{\gamma}             & \assign \exp\ofs{-\frac{i}{2\varepsilon^2}\tau V\ofs{\gamma}}
\end{split}
\end{equation}

where $\widetilde{\omega} \assign \frac{\omega}{n}$ and $\omega \in \widehat{\Gamma}$
when transformed to discretized space.

\subsection{Pseudo code}

With the results from the last sections we are ready to write down a pseudo code for the time
propagation of a $\Ket{\psi}$ which solves the scalar semiclassical Schrödinger
equation \eqref{eq:basics_tdse_semi}. The algorithm \ref{al:Fourier_propagation_single}
shows a straight forward implementation of \eqref{eq:Fourier_propagation_alg}. The
exponentials of the operators are evaluated and stored in a vectorized fashion
simultaneously for all grid nodes $\gamma_i$.

\begin{algorithm}
\caption{Time propagation with operator splitting of $H$ for $\Ket{\psi}$}
\label{al:Fourier_propagation_single}
\begin{algorithmic}
  \REQUIRE The precalculated operator exponentials from \eqref{eq:def_Fourier_operator_exponentials}
  \STATE // Propagate with the potential $V$
  \STATE $\psi^{\prime} \assign V_e \cdot \psi$
  \STATE // Fourier transform
  \STATE $\widehat{\psi}^{\prime} \assign \ft{\psi^{\prime}}$
  \STATE // Propagate with the kinetic operator $T$
  \STATE $\widehat{\psi}^{\prime\prime} \assign T_e \cdot \widehat{\psi}^{\prime}$
  \STATE // Apply inverse Fourier transform
  \STATE $\psi^{\prime\prime} \assign \ift{\widehat{\psi}^{\prime\prime}}$
  \STATE // Propagate again with the potential $V$
  \STATE $\psi^{\left(k+1\right)} \assign V_e \cdot \psi^{\prime\prime}$
  \RETURN $\psi^{\left(k+1\right)}$
\end{algorithmic}
\end{algorithm}


\section{Vector valued states}

In the last section we only considered the simpler case with a potential function
and a scalar state $\Ket{\psi}$. Now we want to extend the theory to matrix
valued potentials like \eqref{eq:general_potential_matrix} together with vector
valued states $\Ket{\Psi}$ defined by \eqref{eq:basics_tdse_vector}. The goal of
this effort is to get an extended version of algorithm \ref{al:Fourier_propagation_single}
that handles this more general case.

The question is now which parts we have to generalize. Because we defined the
Lie-Trotter splitting in an abstract context, the formula \eqref{eq:operator:splitting_propagation}
stays the same. The only things that change are the exponentials therein that become
now matrix exponentials. Hence we have to derive new formulae analogous to the ones
of definition \eqref{eq:def_Fourier_operator_exponentials}.

With the generalized definitions of $T_e$ and $V_e$ the core of the time propagation
algorithm as given in \eqref{eq:Fourier_propagation_alg} is still valid and can be reused.


\subsection{Propagation operator exponentials}

Assume our state $\Ket{\Psi}$ consists of $N$ components $\varphi_0, \ldots, \varphi_{N-1}$.
Thus the Hamiltonian operator $\mathbf{H}$ has to be a $N \times N$ matrix
\footnote{Don't confuse this with the matrix representation $H_{i,j} \assign \Braket{\phi_i | H | \phi_j}$
of a Hamiltonian operator $H$ for a given set of basis functions $\phi_0, \ldots, \phi_k$.}.
We may split $\mathbf{H} = \mathbf{T} + \mathbf{V}$ and write

\begin{equation}
  \mathbf{H} =
  \left(
  \begin{array}{ccc}
    T_0  & {}     & {} \\
    {}   & \ddots & {} \\
    {}   & {}     & T_{N-1} \\
  \end{array} \right)
  +
  \left(
  \begin{array}{ccc}
    v_{0,0}\ofs{x} & \cdots & v_{0,N-1}\ofs{x} \\
    \vdots         &        & \vdots \\
    v_{N-1,0}\ofs{x} & \cdots & v_{N-1,N-1}\ofs{x}
  \end{array} \right)
\end{equation}

where we used the definition of $T$ given by \eqref{eq:basics_def_ops} and the
potential matrix introduced in \eqref{eq:general_potential_matrix}. We can simplify
the first matrix by assuming that all the $T_i$ are identical operators.

With this last step, the exponential $e^{-\frac{i}{\varepsilon^2} \tau \mathbf{T}}$
of the diagonal kinetic operator matrix $\mathbf{T}$ becomes rather easy and the
problem reduces to what we did in the last section. The solution for a single
component is given by $T_e$ of \eqref{eq:def_Fourier_operator_exponentials}.
Therefore we get

\begin{equation} \label{eq:def_Fourier_operator_exponentials_multi_T}
  \mathbf{T_e} \assign
  \exp \ofs{ -\frac{i}{\varepsilon^2} \tau \mathbf{T} }
  =
  \left(
  \begin{array}{ccc}
    T_e  & {}     & {} \\
    {}   & \ddots & {} \\
    {}   & {}     & T_e \\
  \end{array} \right)
\end{equation}

For the potential operator matrix the process is much more difficult as $V\ofs{x}$
is in general not diagonal. Here we really need full-fledged matrix exponentials for
$e^{-\frac{i}{2\varepsilon^2} \tau V\ofs{x}}$ without any possibilities for simplification.
For the special case where $N = 2$ we can use the analytical formula for the
matrix exponential given by \eqref{eq:symbolic_matrix_exp} and adapt it for symmetric
matrices. Otherwise where $N > 2$ numerical techniques have to be used. For the
sake of completeness we note:

\begin{equation} \label{eq:def_Fourier_operator_exponentials_multi_V}
  \mathbf{V_e} \assign
  \exp \ofs{ -\frac{i}{2\varepsilon^2} \tau V\ofs{x} }
  =
  \exp \ofs{ -\frac{i}{2\varepsilon^2} \tau
  \left(
  \begin{array}{ccc}
    {} & {} & {} \\
    {} & V\ofs{x}  & {} \\
    {} & {} & {}
  \end{array} \right)
  }
\end{equation}

Both, $\mathbf{T_e}$ and $\mathbf{V_e}$ are again matrices of the same dimension
as $\mathbf{H}$. Putting all the parts together we arrive at

\begin{equation} \label{eq:Fourier_propagation_alg_multi}
  \Psi_{n+1}\ofs{x} =
  \mathbf{V_e}\ofs{x} \mathcal{F}^{-1} \Bigl(
    \mathbf{T_e}\ofs{\omega} \underbrace{ \mathcal{F}\bigl(
      \mathbf{V_e}\ofs{x} \cdot \Psi_n\ofs{x}
    \bigr)}_{\widehat{\Psi}_n\ofs{\omega}}
  \Bigr)
\end{equation}

which is a time-stepping scheme analogous to \eqref{eq:Fourier_propagation_alg}
but applicable to the Schrödinger equation \eqref{eq:basics_tdse_vector} with
vectorial states $\Ket{\Psi}$.


\subsection{Pseudo code}

With the last result we can start to write down a pseudo code for the time
propagation of a $\Ket{\Psi}$. The algorithm \ref{al:Fourier_propagation_multi}
shows a straight forward implementation of \eqref{eq:Fourier_propagation_alg_multi}.
The values for $\mathbf{T_e}$ and $\mathbf{V_e}$ are precomputed for all
grid nodes $\gamma_i$. We need to be careful with the formal matrix multiplications
this time. Of course we do not build the matrix for $\mathbf{T_e}$ but rather
multiply by the same $T_e$ all the time. This is equivalent to the redefinition
$\mathbf{T_e} \assign T_e$.

\begin{algorithm}
\caption{Time propagation with operator splitting of $H$ for $\Ket{\Psi}$}
\label{al:Fourier_propagation_multi}
\begin{algorithmic}
  \REQUIRE The precalculated operator exponentials from \eqref{eq:def_Fourier_operator_exponentials_multi_T} and \eqref{eq:def_Fourier_operator_exponentials_multi_V}
  \STATE $\Psi^{\left(k\right)} = \{ \varphi_0, \ldots, \varphi_{N-1} \}$

  \STATE // Propagate with the potential $V$
  \STATE $\Psi^{\prime} \assign \{0, \ldots, 0\}$
  \FOR{ $r \assign 0$ \TO $N-1$}
    \FOR{ $c \assign 0$ \TO $N-1$}
      \STATE $\Psi^{\prime}_r \assign \Psi^{\prime}_r + \mathbf{Ve}_{r,c} \cdot \Psi^{\left(k\right)}_c$
    \ENDFOR
  \ENDFOR

  \STATE // Fourier transform the components
  \STATE $\widehat{\Psi}^{\prime}  \assign \{0, \ldots, 0\}$
  \FOR{ $r \assign 0$ \TO $N-1$}
    \STATE $\widehat{\Psi}^{\prime}_r \assign \ft{\Psi^{\prime}_r}$
  \ENDFOR

  \STATE // Propagate with the kinetic operator $T$
  \FOR{ $r \assign 0$ \TO $N-1$}
    \STATE $\widehat{\Psi}^{\prime\prime}_r \assign \mathbf{T_e} \cdot \widehat{\Psi}^{\prime}_r$
  \ENDFOR

  \STATE // Apply inverse Fourier transform to the components
  \STATE $\Psi^{\prime\prime}  \assign \{0, \ldots, 0\}$
  \FOR{ $r \assign 0$ \TO $N-1$}
    \STATE $\Psi^{\prime\prime}_r \assign \ift{\widehat{\Psi}^{\prime\prime}_r}$
  \ENDFOR

  \STATE // Propagate again with the potential $V$
  \STATE $\Psi^{\left(k+1\right)} \assign \{0, \ldots, 0\}$
  \FOR{ $r \assign 0$ \TO $N-1$}
    \FOR{ $c \assign 0$ \TO $N-1$}
      \STATE $\Psi^{\left(k+1\right)}_r \assign \Psi^{\left(k+1\right)}_r + \mathbf{Ve}_{r,c} \cdot \Psi^{\prime\prime}_c$
    \ENDFOR
  \ENDFOR
  \RETURN $\Psi^{\left(k+1\right)}$
\end{algorithmic}
\end{algorithm}


\section{Initial values}

We just finished building an algorithm that can perform the time evolution of any
given initial state. It's time to pay attention to these initial values and elaborate
how we have to define them properly.

In the case of multiple energy levels $\lambda_i$ we need to specify the initial
values with extra care. In most cases we want to start with a single Gaussian
wavepacket on only one energy level and nothing on the others. This Gaussian
wavepacket we start with is localized around position $q$ and may have a momentum
that is localized in momentum space around $p$. For an example how this
could look like see figure \ref{fig:initial_values_fourier} where we start with
a right travelling packet on the left side of the avoided crossing.

\begin{figure}
  \centering
  \begin{asy}[width=\the\linewidth]
import graph;

real left = -5;
real right = 5;

real delta = 0.15;

real lambda0(real x) { return sqrt(tanh(x)^2+delta**2)/2; }
real lambda1(real x) { return -sqrt(tanh(x)^2+delta**2)/2; }

xaxis("$x$",Arrow);
yaxis("$y$",Arrow);

draw(graph(lambda0,left,right,operator ..),blue);
draw(graph(lambda1,left,right,operator ..),blue);

real u = lambda0(2);
pair l1 = (1,u);

real u = lambda1(2);
pair l0 = (1,u);

label("$\lambda_1(x)$",l1+1.5*S);
label("$\lambda_0(x)$",l0+1.5*N);

real gauss(real x) {
  real mu = -3.5;
  real sigma = 0.09;
  real gamma = 0.2;
  return lambda0(mu) + gamma * 1.0/sqrt(2.*pi * sigma**2) * exp(-(x-mu)**2/(2.0*sigma**2));
}

draw(graph(gauss,-3.8,-3.2,500,operator --),orange);

real s = lambda0(-3.5) + 0.3;
pair s = (-3.5,s);

draw(s--(s+0.5*E),orange,Arrow,DotMargin);
dot(s,orange);
label("$p$",s+0.25*E+0.3N,orange);

pair s = (-3.5,0);
dot(s, orange);
label("$q$",s+0.25*N,orange);
\end{asy}

  \caption{Example of Gaussian initial values on the upper energy level.}
  \label{fig:initial_values_fourier}
\end{figure}

It's apparent that the initial values are given in the eigenbasis of the potential
$V$. But the simulation takes place in the canonical basis thus the initial values
have to be transformed. This happens with a simple linear basis transformation
from the eigenbasis into the canonical basis. The orthogonal matrix $M$ that performs
this task is given by the eigenvectors of our potential. We defined this
transformation together with the inverse in the section \ref{sec:basis_transformations_of_states}.

As starting point of the time-stepping algorithm \eqref{al:Fourier_propagation_multi}
we write now for the values $\Psi_0\ofs{x}$ just before our iteration first takes place

\begin{equation}
  \Psi_0\ofs{x} \assign M \Ket{\Psi_{\text{IV}}}
\end{equation}

where $\Ket{\Psi_{\text{IV}}}$ are the given initial values in the eigenbasis. Even if
we start in most situations with a single Gaussian packet, the initial values can be
arbitrary and different on each energy surface. In general they don't have to be
localized wavepackets at all.


\section{Observables}

In this section we will look into the calculation of observables like the different
energies and norms. Note that we always use the complex scalar product $\dotp{\cdot}{\cdot}$
here.

\subsection{The norm of a wavepacket}

The norm of a wavepacket is particularly interesting because it can be interpreted
as the probability density for finding the particle in a specified infinitesimal volume
element.

To calculate the norm of a wavepacket $\Ket{\varPhi}$ we start from the definition
and use again a transformation to Fourier space and Parseval's identity. This gives us

\begin{equation}
  \| \Psi\ofs{x} \|_{L^2}^2 \assign \Braket{\Psi | \Psi}
  = \Braket{\widehat{\Psi} | \widehat{\Psi}}
  = \| \widehat{\Psi}\ofs{\omega} \|_{L^2}^2
\end{equation}

In the discretized space we evaluate $\Psi$ at the grid nodes $\Gamma$ and get a vector with
$n$ entries.

\begin{equation}
\begin{split}
  \| \Psi\ofs{\Gamma} \|_{2}^2 & = \dotp{\Psi\ofs{\Gamma}}{\Psi\ofs{\Gamma}} \\
                               & = \sum_i^n \conj{\Psi\ofs{\gamma_i}} \Psi\ofs{\gamma_i} \\
                               & = \frac{2 \pi f}{n^2} \sum_k^n \conj{\widehat{\Psi}\ofs{\omega_k}} \widehat{\Psi}\ofs{\omega_k} \\
                               & = \frac{2 \pi f}{n^2} \dotp{\widehat{\Psi}\ofs{\omega_k}}{\widehat{\Psi}\ofs{\omega_k}} \\
                               & = \| \widehat{\Psi}(\widehat{\Gamma}) \|_{2}^2
\end{split}
\end{equation}

where we used a discrete Fourier transformation.
Further, with the square removed by taking the root, we get

\begin{equation}
  \| \widehat{\Psi}(\widehat{\Gamma}) \|_{2}
  = \frac{\sqrt{2 \pi f}}{n} \| \widehat{\Psi}\ofs{\omega} \|_{2}
\end{equation}

This is the formula we use in the code because it's much cheaper to calculate the
norm in Fourier space.

Notice that for the case of vector valued wave functions the norm of a single component $\varphi_i$
is exactly the probability for finding the particle on the corresponding energy level $E_i$.


\subsection{Energy of a wavepacket}

The energy of a quantum wavepacket $\Ket{\Psi}$ is given by the following integral

\begin{equation}
  E = \Braket{\Psi | H | \Psi}
\end{equation}

Hence we investigate the detailed structure of this expression with the aim to easily
calculate it's value. First we explicitly split the expression into the parts
for potential and kinetic energy

\begin{equation} \label{eq:energy_split_Fourier}
  \Braket{\Psi | H | \Psi} = \Braket{\Psi | T | \Psi} + \Braket{\Psi | V | \Psi}
\end{equation}

and then we plug in the operators' definitions given by \eqref{eq:basics_def_ops}.
This yields

\begin{equation}
  \Braket{\Psi | H | \Psi} = \Braket{\Psi | -\frac{1}{2} \varepsilon^4 \frac{\partial^2}{\partial x^2} | \Psi}
                           + \Braket{\Psi | V\ofs{x} | \Psi}
\end{equation}

The first summand which represents the kinetic energy $E_{\text{kin}}$ can be simplified
using linearity

\begin{align*}
  E_{\text{kin}} & = -\frac{1}{2} \varepsilon^4 \Braket{\Psi | \frac{\partial^2}{\partial x^2} | \Psi} \\
  \intertext{and by transformation to Fourier space}
                 & = -\frac{1}{2} \varepsilon^4 \Braket{\ft{\Psi} | \left(i \omega\right)^2 | \ft{\Psi}} \\
                 & = \frac{1}{2} \varepsilon^4 \Braket{\Psi\ofs{\omega} | \omega^2 | \Psi\ofs{\omega}}
\end{align*}

Finally we get the following formula for the kinetic energy of a wavepacket $\Ket{\Psi}$

\begin{equation} \label{eq:Fourier_ekin_scalar}
  E_{\text{kin}} = \frac{1}{2} \varepsilon^4 \int_\omega \conj{\widehat{\Psi}\ofs{\omega}} \cdot \omega^2 \cdot \widehat{\Psi}\ofs{\omega} d\omega
\end{equation}

While we were able to simplify the expression for the kinetic energy quite a lot
this is not possible in the same manner for a general potential $V\ofs{x}$.Thus
we just apply Parseval's equality and write

\begin{align} \label{eq:Fourier_epot_scalar}
  E_{\text{pot}} & = \Braket{\Psi | V\ofs{x} | \Psi} \nonumber \\
                 & = \Braket{\ft{\Psi\ofs{x}} | \ft{V\ofs{x} \Psi\ofs{x}}} \nonumber \\
                 & = \int_\omega \ft{\Psi\ofs{x}} \ft{V\ofs{x} \Psi\ofs{x}} d\omega
\end{align}


\subsection{Energy of a vector valued wavepacket}

In the case of vector valued wavepackets $\Ket{\Psi}$
we have to think carefully what to compute. The basic equation \eqref{eq:energy_split_Fourier}
is still valid but we have to use the matrix valued operators $\mathbf{T}$ and $\mathbf{V}$

\begin{equation}
  E =  \Braket{\Psi | \mathbf{T} | \Psi } + \Braket{\Psi | \mathbf{V} | \Psi }
\end{equation}

The computation of both energies has to be carried out in the canonical basis because
we need the operators $\mathbf{T}$ and $\mathbf{V}$ whose representations we only
know there. On the other hand we are interested in the energy of the wavepacket $\Ket{\Psi}$
and its components $\varphi_i$ measured in the eigenbasis. Thus there is an additional
basis transformation $M$ involved which drops out in the scalar case.
Of course the overall energy is independent of any particular basis, but the energy
of a component $\varphi_i$ is not. In analogy to \eqref{eq:Fourier_ekin_scalar} and
\eqref{eq:Fourier_epot_scalar} we can write

\begin{equation}
  E_{\text{kin}} =
  \Braket{
    \begin{pmatrix}
      \varphi_0 \\
      \vdots \\
      \varphi_{N-1}
    \end{pmatrix}
    |
    M\T
    \begin{pmatrix}
      T  & {}     & {} \\
      {} & \ddots & {} \\
      {} & {}     & T  \\
    \end{pmatrix}
    M
    |
    \begin{pmatrix}
      \varphi_0 \\
      \vdots \\
      \varphi_{N-1}
    \end{pmatrix}
  }
\end{equation}

and

\begin{equation}
  E_{\text{pot}} =
  \Braket{
    \begin{pmatrix}
      \varphi_0 \\
      \vdots \\
      \varphi_{N-1}
    \end{pmatrix}
    |
    M\T
    \begin{pmatrix}
      {} & {}       & {} \\
      {} & V\ofs{x} & {} \\
      {} & {}       & {} \\
    \end{pmatrix}
    M
    |
    \begin{pmatrix}
      \varphi_0 \\
      \vdots \\
      \varphi_{N-1}
    \end{pmatrix}
  }
\end{equation}

For the calculation of the energy of a single component $\varphi_i$
of $\Ket{\Psi}$ we transform and measure according to this formula

\begin{equation}
  E_{\text{pot}}^i =
  \Braket{
    \begin{pmatrix}
      0 \\
      \varphi_i \\
      0
    \end{pmatrix}
    |
    M\T
    \mathbf{V}
    M
    |
    \begin{pmatrix}
      0 \\
      \varphi_i \\
      0
    \end{pmatrix}
  }
\end{equation}

for the potential energy or its identical counterpart with the operator $\mathbf{V}$ replaced
by $\mathbf{T}$ for kinetic energy. In any case it holds that

\begin{equation}
\begin{split}
  E_{\text{total}} & = E_{\text{kin}} + E_{\text{pot}} \\
                   & = \sum_{i=0}^{N-1} E_{\text{kin}}^i + \sum_{i=0}^{N-1} E_{\text{pot}}^i \\
                   & = \text{constant}
\end{split}
\end{equation}

We have conservation of energy as the system is self-contained.

\subsection{Energy computations in discretized space}

For the energies we get within discretized space the following formulae where we
again build a vector from evaluating the function $\psi$ on the grid nodes $\Gamma$:

\begin{align} \label{eq:dicrete_energies}
  E_{\text{kin}} & = \frac{2 \pi f}{n^2} \frac{1}{2} \varepsilon^4 \dotp{\conj{\ft{\psi(\Gamma)}}}{{\omega^2 \ft{\psi(\Gamma)}}} \\
  \intertext{and}
  E_{\text{pot}} & = \frac{2 \pi f}{n^2} \dotp{\conj{\ft{\psi(\Gamma)}}}{{V\ofs{x} \ft{\psi(\Gamma)}}}
\end{align}

For a vectorial wavepacket $\Ket{\Psi}$ we have

\begin{equation}
  E_{\text{pot}} = \frac{2 \pi f}{n^2} \dotp{\ft{\varphi_i}}{\ft{\widetilde{\varphi_i}}}
\end{equation}

where $\left(\widetilde{\varphi}_0, \ldots, \widetilde{\varphi}_{N-1} \right)\T \assign \widetilde{\Psi} = \mathbf{V} \Psi$
for the potential energy. To calculate the kinetic energy we apply the above formula \eqref{eq:dicrete_energies}
to each component $\psi_i$ of $\Psi$ separately.

Of course the Fourier transformation $\ft{\cdot}$ has to be interpreted as a discrete
Fourier transformation. The discrete Fourier transformation is then implemented as
an \texttt{FFT} algorithm.
