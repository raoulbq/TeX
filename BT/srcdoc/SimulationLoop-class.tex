%%%%%%%%%%%%%%%%%%%%%%%%%%%%%%%%%%%%%%%%%%%%%%%%%%%%%%%%%%%%%%%%%%%%%%%%%%%
%%                           Class Description                           %%
%%%%%%%%%%%%%%%%%%%%%%%%%%%%%%%%%%%%%%%%%%%%%%%%%%%%%%%%%%%%%%%%%%%%%%%%%%%

    \index{SimulationLoop \textit{(module)}!SimulationLoop \textit{(class)}|(}
\section{Class SimulationLoop}

    \label{SimulationLoop:SimulationLoop}
This class acts as the main simulation loop. It owns a propagator that
propagates a set of initial values during a time evolution. All values are
read from the \textit{Parameters.py} file.


%%%%%%%%%%%%%%%%%%%%%%%%%%%%%%%%%%%%%%%%%%%%%%%%%%%%%%%%%%%%%%%%%%%%%%%%%%%
%%                                Methods                                %%
%%%%%%%%%%%%%%%%%%%%%%%%%%%%%%%%%%%%%%%%%%%%%%%%%%%%%%%%%%%%%%%%%%%%%%%%%%%

  \subsection{Methods}

    \label{SimulationLoop:SimulationLoop:__init__}
    \index{SimulationLoop \textit{(module)}!SimulationLoop \textit{(class)}!SimulationLoop.\_\_init\_\_ \textit{(method)}}

    \vspace{0.5ex}

\hspace{.8\funcindent}\begin{boxedminipage}{\funcwidth}

    \raggedright \textbf{\_\_init\_\_}(\textit{self})

    \vspace{-1.5ex}

    \rule{\textwidth}{0.5\fboxrule}
\setlength{\parskip}{2ex}
    Create a new simulation loop instance.

\setlength{\parskip}{1ex}
    \end{boxedminipage}

    \label{SimulationLoop:SimulationLoop:add_fourier_propagator}
    \index{SimulationLoop \textit{(module)}!SimulationLoop \textit{(class)}!SimulationLoop.add\_fourier\_propagator \textit{(method)}}

    \vspace{0.5ex}

\hspace{.8\funcindent}\begin{boxedminipage}{\funcwidth}

    \raggedright \textbf{add\_fourier\_propagator}(\textit{self})

    \vspace{-1.5ex}

    \rule{\textwidth}{0.5\fboxrule}
\setlength{\parskip}{2ex}
    Set up a Fourier propagator for the simulation loop. Set the potential
    and initial values according to the configuration.

\setlength{\parskip}{1ex}
      \textbf{Raises}
    \vspace{-1ex}

      \begin{quote}
        \begin{description}

          \item[\texttt{ValueError}]

          For invalid or missing input data.

        \end{description}

      \end{quote}

    \end{boxedminipage}

    \label{SimulationLoop:SimulationLoop:add_hagedorn_propagator}
    \index{SimulationLoop \textit{(module)}!SimulationLoop \textit{(class)}!SimulationLoop.add\_hagedorn\_propagator \textit{(method)}}

    \vspace{0.5ex}

\hspace{.8\funcindent}\begin{boxedminipage}{\funcwidth}

    \raggedright \textbf{add\_hagedorn\_propagator}(\textit{self})

    \vspace{-1.5ex}

    \rule{\textwidth}{0.5\fboxrule}
\setlength{\parskip}{2ex}
    Set up a Hagedorn propagator for the simulation loop. Set the potential
    and initial values according to the configuration.

\setlength{\parskip}{1ex}
      \textbf{Raises}
    \vspace{-1ex}

      \begin{quote}
        \begin{description}

          \item[\texttt{ValueError}]

          For invalid or missing input data.

        \end{description}

      \end{quote}

    \end{boxedminipage}

    \label{SimulationLoop:SimulationLoop:add_multi_hagedorn_propagator}
    \index{SimulationLoop \textit{(module)}!SimulationLoop \textit{(class)}!SimulationLoop.add\_multi\_hagedorn\_propagator \textit{(method)}}

    \vspace{0.5ex}

\hspace{.8\funcindent}\begin{boxedminipage}{\funcwidth}

    \raggedright \textbf{add\_multi\_hagedorn\_propagator}(\textit{self})

    \vspace{-1.5ex}

    \rule{\textwidth}{0.5\fboxrule}
\setlength{\parskip}{2ex}
    Set up a multi Hagedorn propagator for the simulation loop. Set the
    potential and initial values according to the configuration.

\setlength{\parskip}{1ex}
      \textbf{Raises}
    \vspace{-1ex}

      \begin{quote}
        \begin{description}

          \item[\texttt{ValueError}]

          For invalid or missing input data.

        \end{description}

      \end{quote}

    \end{boxedminipage}

    \label{SimulationLoop:SimulationLoop:run_fourier_propagator}
    \index{SimulationLoop \textit{(module)}!SimulationLoop \textit{(class)}!SimulationLoop.run\_fourier\_propagator \textit{(method)}}

    \vspace{0.5ex}

\hspace{.8\funcindent}\begin{boxedminipage}{\funcwidth}

    \raggedright \textbf{run\_fourier\_propagator}(\textit{self})

    \vspace{-1.5ex}

    \rule{\textwidth}{0.5\fboxrule}
\setlength{\parskip}{2ex}
    Run the simulation loop for a number of time steps. The number of steps
    is calculated in the \textit{initialize} function.

\setlength{\parskip}{1ex}
    \end{boxedminipage}

    \label{SimulationLoop:SimulationLoop:run_hagedorn_propagator}
    \index{SimulationLoop \textit{(module)}!SimulationLoop \textit{(class)}!SimulationLoop.run\_hagedorn\_propagator \textit{(method)}}

    \vspace{0.5ex}

\hspace{.8\funcindent}\begin{boxedminipage}{\funcwidth}

    \raggedright \textbf{run\_hagedorn\_propagator}(\textit{self})

    \vspace{-1.5ex}

    \rule{\textwidth}{0.5\fboxrule}
\setlength{\parskip}{2ex}
    Run the simulation loop for a number of time steps. The number of steps
    is calculated in the \textit{initialize} function.

\setlength{\parskip}{1ex}
    \end{boxedminipage}

    \label{SimulationLoop:SimulationLoop:run_multi_hagedorn_propagator}
    \index{SimulationLoop \textit{(module)}!SimulationLoop \textit{(class)}!SimulationLoop.run\_multi\_hagedorn\_propagator \textit{(method)}}

    \vspace{0.5ex}

\hspace{.8\funcindent}\begin{boxedminipage}{\funcwidth}

    \raggedright \textbf{run\_multi\_hagedorn\_propagator}(\textit{self})

    \vspace{-1.5ex}

    \rule{\textwidth}{0.5\fboxrule}
\setlength{\parskip}{2ex}
    Run the simulation loop for a number of time steps. The number of steps
    is calculated in the \textit{initialize} function.

\setlength{\parskip}{1ex}
    \end{boxedminipage}

    \label{SimulationLoop:SimulationLoop:end_simulation}
    \index{SimulationLoop \textit{(module)}!SimulationLoop \textit{(class)}!SimulationLoop.end\_simulation \textit{(method)}}

    \vspace{0.5ex}

\hspace{.8\funcindent}\begin{boxedminipage}{\funcwidth}

    \raggedright \textbf{end\_simulation}(\textit{self})

    \vspace{-1.5ex}

    \rule{\textwidth}{0.5\fboxrule}
\setlength{\parskip}{2ex}
    Do the necessary cleanup after a simulation. For example request the
    serializer to write the data and close the output files.

\setlength{\parskip}{1ex}
    \end{boxedminipage}


%%%%%%%%%%%%%%%%%%%%%%%%%%%%%%%%%%%%%%%%%%%%%%%%%%%%%%%%%%%%%%%%%%%%%%%%%%%
%%                          Instance Variables                           %%
%%%%%%%%%%%%%%%%%%%%%%%%%%%%%%%%%%%%%%%%%%%%%%%%%%%%%%%%%%%%%%%%%%%%%%%%%%%

  \subsection{Instance Variables}

    \vspace{-1cm}
\hspace{\varindent}\begin{longtable}{|p{\varnamewidth}|p{\vardescrwidth}|l}
\cline{1-2}
\cline{1-2} \centering \textbf{Name} & \centering \textbf{Description}& \\
\cline{1-2}
\endhead\cline{1-2}\multicolumn{3}{r}{\small\textit{continued on next page}}\\\endfoot\cline{1-2}
\endlastfoot\raggedright p\-r\-o\-p\-a\-g\-a\-t\-o\-r\- & The time propagator instance driving the simulation.&\\
\cline{1-2}
\raggedright s\-e\-r\-i\-a\-l\-i\-z\-e\-r\- & A \textit{Serializer} instance for saving simulation results.&\\
\cline{1-2}
\raggedright n\-s\-t\-e\-p\-s\- & The number of time steps we will perform.&\\
\cline{1-2}
\end{longtable}

    \index{SimulationLoop \textit{(module)}!SimulationLoop \textit{(class)}|)}
