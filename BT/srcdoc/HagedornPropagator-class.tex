%%%%%%%%%%%%%%%%%%%%%%%%%%%%%%%%%%%%%%%%%%%%%%%%%%%%%%%%%%%%%%%%%%%%%%%%%%%
%%                           Class Description                           %%
%%%%%%%%%%%%%%%%%%%%%%%%%%%%%%%%%%%%%%%%%%%%%%%%%%%%%%%%%%%%%%%%%%%%%%%%%%%

    \index{HagedornPropagator \textit{(module)}!HagedornPropagator \textit{(class)}|(}
\section{Class HagedornPropagator}

    \label{HagedornPropagator:HagedornPropagator}
\begin{tabular}{cccccc}
% Line for Propagator, linespec=[False]
\multicolumn{2}{r}{\settowidth{\BCL}{Propagator}\multirow{2}{\BCL}{Propagator}}
&&
  \\\cline{3-3}
  &&\multicolumn{1}{c|}{}
&&
  \\
&&\multicolumn{2}{l}{\textbf{HagedornPropagator}}
\end{tabular}

This class can numerically propagate given initial values
$\Ket{\Psi}$ in a potential
$V\ofs{x}$. The propagation is done for a given
homogeneous Hagedorn wavepacket.


%%%%%%%%%%%%%%%%%%%%%%%%%%%%%%%%%%%%%%%%%%%%%%%%%%%%%%%%%%%%%%%%%%%%%%%%%%%
%%                                Methods                                %%
%%%%%%%%%%%%%%%%%%%%%%%%%%%%%%%%%%%%%%%%%%%%%%%%%%%%%%%%%%%%%%%%%%%%%%%%%%%

  \subsection{Methods}

    \vspace{0.5ex}

\hspace{.8\funcindent}\begin{boxedminipage}{\funcwidth}

    \raggedright \textbf{\_\_init\_\_}(\textit{self}, \textit{potential}, \textit{packet}, \textit{leading\_component})

    \vspace{-1.5ex}

    \rule{\textwidth}{0.5\fboxrule}
\setlength{\parskip}{2ex}
    Initialize a new \textit{HagedornPropagator} instance.

\setlength{\parskip}{1ex}
      \textbf{Parameters}
      \vspace{-1ex}

      \begin{quote}
        \begin{Ventry}{xxxxxxxxxxxxxxxxx}

          \item[potential]

          The potential the wavepacket
          $\Ket{\Psi}$ feels during the
          time propagation.

          \item[packet]

          The initial homogeneous Hagedorn wavepacket we propagate in time.

          \item[leading\_component]

          The leading component index $\chi$.

        \end{Ventry}

      \end{quote}

      \textbf{Raises}
    \vspace{-1ex}

      \begin{quote}
        \begin{description}

          \item[\texttt{ValueError}]

          If the number of components of
          $\Ket{\Psi}$ does not match the
          number of energy levels $\lambda_i$ of the
          potential.

        \end{description}

      \end{quote}

      Overrides: Propagator.\_\_init\_\_

    \end{boxedminipage}

    \vspace{0.5ex}

\hspace{.8\funcindent}\begin{boxedminipage}{\funcwidth}

    \raggedright \textbf{\_\_str\_\_}(\textit{self})

    \vspace{-1.5ex}

    \rule{\textwidth}{0.5\fboxrule}
\setlength{\parskip}{2ex}
    Prepare a printable string representing the \textit{HagedornPropagator}
    instance.

\setlength{\parskip}{1ex}

      Overrides: Propagator.\_\_str\_\_

    \end{boxedminipage}

    \vspace{0.5ex}

\hspace{.8\funcindent}\begin{boxedminipage}{\funcwidth}

    \raggedright \textbf{get\_number\_components}(\textit{self})

    \vspace{-1.5ex}

    \rule{\textwidth}{0.5\fboxrule}
\setlength{\parskip}{2ex}
\setlength{\parskip}{1ex}
      \textbf{Return Value}
    \vspace{-1ex}

      \begin{quote}
      The number $N$ of components $\Phi_i$ of
      $\Ket{\Psi}$.

      \end{quote}

      Overrides: Propagator.get\_number\_components

    \end{boxedminipage}

    \vspace{0.5ex}

\hspace{.8\funcindent}\begin{boxedminipage}{\funcwidth}

    \raggedright \textbf{get\_potential}(\textit{self})

    \vspace{-1.5ex}

    \rule{\textwidth}{0.5\fboxrule}
\setlength{\parskip}{2ex}
\setlength{\parskip}{1ex}
      \textbf{Return Value}
    \vspace{-1ex}

      \begin{quote}
      The \textit{MatrixPotential} instance used for time propagation.

      \end{quote}

      Overrides: Propagator.get\_potential

    \end{boxedminipage}

    \label{HagedornPropagator:HagedornPropagator:get_wavepacket}
    \index{HagedornPropagator \textit{(module)}!HagedornPropagator \textit{(class)}!HagedornPropagator.get\_wavepacket \textit{(method)}}

    \vspace{0.5ex}

\hspace{.8\funcindent}\begin{boxedminipage}{\funcwidth}

    \raggedright \textbf{get\_wavepacket}(\textit{self})

    \vspace{-1.5ex}

    \rule{\textwidth}{0.5\fboxrule}
\setlength{\parskip}{2ex}
\setlength{\parskip}{1ex}
      \textbf{Return Value}
    \vspace{-1ex}

      \begin{quote}
      The \textit{HagedornWavepacket} instance that represents the current
      wavepacket $\Ket{\Psi}$.

      \end{quote}

    \end{boxedminipage}

    \vspace{0.5ex}

\hspace{.8\funcindent}\begin{boxedminipage}{\funcwidth}

    \raggedright \textbf{get\_wavefunction}(\textit{self}, \textit{nodes})

    \vspace{-1.5ex}

    \rule{\textwidth}{0.5\fboxrule}
\setlength{\parskip}{2ex}
    Construct a \textit{WaveFunction} object which contains the components
    $\Phi_i$ of the Hagedorn wavepacket evaluated at the
    given nodes $\gamma$.

\setlength{\parskip}{1ex}
      \textbf{Parameters}
      \vspace{-1ex}

      \begin{quote}
        \begin{Ventry}{xxxxx}

          \item[nodes]

          The nodes $\gamma$ on which the Hagedorn
          wavepacket is evaluated.

        \end{Ventry}

      \end{quote}

      \textbf{Return Value}
    \vspace{-1ex}

      \begin{quote}
      A \textit{WaveFunction} instance representing the values of the
      current $\Ket{\Psi}$.

      \end{quote}

\textbf{Note:} This method is quite expensive.

      Overrides: Propagator.get\_wavefunction

    \end{boxedminipage}

    \vspace{0.5ex}

\hspace{.8\funcindent}\begin{boxedminipage}{\funcwidth}

    \raggedright \textbf{propagate}(\textit{self})

    \vspace{-1.5ex}

    \rule{\textwidth}{0.5\fboxrule}
\setlength{\parskip}{2ex}
    Given the wavepacket $\Psi$ at time $t$, calculate a
    new wavepacket at time $t + \tau$. We perform exactly one timestep
    $\tau$ here.

\setlength{\parskip}{1ex}

      Overrides: Propagator.propagate

    \end{boxedminipage}

%%%%%%%%%%%%%%%%%%%%%%%%%%%%%%%%%%%%%%%%%%%%%%%%%%%%%%%%%%%%%%%%%%%%%%%%%%%
%%                          Instance Variables                           %%
%%%%%%%%%%%%%%%%%%%%%%%%%%%%%%%%%%%%%%%%%%%%%%%%%%%%%%%%%%%%%%%%%%%%%%%%%%%

  \subsection{Instance Variables}

    \vspace{-1cm}
\hspace{\varindent}\begin{longtable}{|p{\varnamewidth}|p{\vardescrwidth}|l}
\cline{1-2}
\cline{1-2} \centering \textbf{Name} & \centering \textbf{Description}& \\
\cline{1-2}
\endhead\cline{1-2}\multicolumn{3}{r}{\small\textit{continued on next page}}\\\endfoot\cline{1-2}
\endlastfoot\raggedright p\-o\-t\-e\-n\-t\-i\-a\-l\- & The potential $V\ofs{x}$ the packet feels.&\\
\cline{1-2}
\raggedright {number\_components} & Number $N$ of components the wavepacket
          $\Ket{\Psi}$ has got.&\\
\cline{1-2}
\raggedright l\-e\-a\-d\-i\-n\-g\- & The leading component $\chi$ is the index of the
          eigenvalue of the potential that is responsible for propagating
          the Hagedorn parameters.&\\
\cline{1-2}
\raggedright p\-a\-c\-k\-e\-t\- & The Hagedorn wavepacket.&\\
\cline{1-2}
\end{longtable}

    \index{HagedornPropagator \textit{(module)}!HagedornPropagator \textit{(class)}|)}
