%%%%%%%%%%%%%%%%%%%%%%%%%%%%%%%%%%%%%%%%%%%%%%%%%%%%%%%%%%%%%%%%%%%%%%%%%%%
%%                           Class Description                           %%
%%%%%%%%%%%%%%%%%%%%%%%%%%%%%%%%%%%%%%%%%%%%%%%%%%%%%%%%%%%%%%%%%%%%%%%%%%%

    \index{MatrixPotential2S \textit{(module)}!MatrixPotential2S \textit{(class)}|(}
\section{Class MatrixPotential2S}

    \label{MatrixPotential2S:MatrixPotential2S}
\begin{tabular}{cccccc}
% Line for MatrixPotential, linespec=[False]
\multicolumn{2}{r}{\settowidth{\BCL}{MatrixPotential}\multirow{2}{\BCL}{MatrixPotential}}
&&
  \\\cline{3-3}
  &&\multicolumn{1}{c|}{}
&&
  \\
&&\multicolumn{2}{l}{\textbf{MatrixPotential2S}}
\end{tabular}

This class represents a matrix potential $V\ofs{x}$. The
potential is given as an analytical $2 \times 2$ matrix expression.
Some symbolical calculations with the potential are supported. For example
calculation of eigenvalues and exponentials and numerical evaluation.
Further, there are methods for splitting the potential into a Taylor
expansion and for basis transformations between canonical and eigenbasis.


%%%%%%%%%%%%%%%%%%%%%%%%%%%%%%%%%%%%%%%%%%%%%%%%%%%%%%%%%%%%%%%%%%%%%%%%%%%
%%                                Methods                                %%
%%%%%%%%%%%%%%%%%%%%%%%%%%%%%%%%%%%%%%%%%%%%%%%%%%%%%%%%%%%%%%%%%%%%%%%%%%%

  \subsection{Methods}

    \vspace{0.5ex}

\hspace{.8\funcindent}\begin{boxedminipage}{\funcwidth}

    \raggedright \textbf{\_\_init\_\_}(\textit{self}, \textit{expression})

    \vspace{-1.5ex}

    \rule{\textwidth}{0.5\fboxrule}
\setlength{\parskip}{2ex}
    Create a new \textit{MatrixPotential2S} instance for a given potential
    matrix $V\ofs{x}$.

\setlength{\parskip}{1ex}
      \textbf{Parameters}
      \vspace{-1ex}

      \begin{quote}
        \begin{Ventry}{xxxxxxxxxx}

          \item[expression]

          An expression representing the potential.

        \end{Ventry}

      \end{quote}



      Overrides: MatrixPotential.\_\_init\_\_

    \end{boxedminipage}

    \vspace{0.5ex}

\hspace{.8\funcindent}\begin{boxedminipage}{\funcwidth}

    \raggedright \textbf{\_\_str\_\_}(\textit{self})

    \vspace{-1.5ex}

    \rule{\textwidth}{0.5\fboxrule}
\setlength{\parskip}{2ex}
    Put the number of components and the analytical expression (the matrix)
    into a printable string.

\setlength{\parskip}{1ex}


      Overrides: MatrixPotential.\_\_str\_\_

    \end{boxedminipage}

    \vspace{0.5ex}

\hspace{.8\funcindent}\begin{boxedminipage}{\funcwidth}

    \raggedright \textbf{get\_number\_components}(\textit{self})

    \vspace{-1.5ex}

    \rule{\textwidth}{0.5\fboxrule}
\setlength{\parskip}{2ex}
\setlength{\parskip}{1ex}
      \textbf{Return Value}
    \vspace{-1ex}

      \begin{quote}
      The number $N$ of components the potential supports. This is also
      the size of the matrix. In the current case it's 2.

      \end{quote}



      Overrides: MatrixPotential.get\_number\_components

    \end{boxedminipage}

    \vspace{0.5ex}

\hspace{.8\funcindent}\begin{boxedminipage}{\funcwidth}

    \raggedright \textbf{evaluate\_at}(\textit{self}, \textit{nodes}, \textit{component}={\tt None}, \textit{as\_matrix}={\tt False})

    \vspace{-1.5ex}

    \rule{\textwidth}{0.5\fboxrule}
\setlength{\parskip}{2ex}
    Evaluate the potential matrix elementwise at some given grid nodes
    $\gamma$.

\setlength{\parskip}{1ex}
      \textbf{Parameters}
      \vspace{-1ex}

      \begin{quote}
        \begin{Ventry}{xxxxxxxxx}

          \item[nodes]

          The grid nodes $\gamma$ we want to evaluate the
          potential at.

          \item[component]

          The component $V_{i,j}$ that gets evaluated or \textit{None} to
          evaluate all.

          \item[as\_matrix]

          Dummy parameter which has no effect here.

        \end{Ventry}

      \end{quote}

      \textbf{Return Value}
    \vspace{-1ex}

      \begin{quote}
      A list with the $4$ entries evaluated at the nodes.

      \end{quote}



      Overrides: MatrixPotential.evaluate\_at

    \end{boxedminipage}

    \vspace{0.5ex}

\hspace{.8\funcindent}\begin{boxedminipage}{\funcwidth}

    \raggedright \textbf{calculate\_eigenvalues}(\textit{self})

    \vspace{-1.5ex}

    \rule{\textwidth}{0.5\fboxrule}
\setlength{\parskip}{2ex}
    Calculate the two eigenvalues
    $\lambda_i\ofs{x}$ of the potential
    $V\ofs{x}$. We can do this by symbolical
    calculations. The multiplicities are taken into account.

\setlength{\parskip}{1ex}
\textbf{Note:} Note: the eigenvalues are memoized for later reuse.


      Overrides: MatrixPotential.calculate\_eigenvalues




    \end{boxedminipage}

    \vspace{0.5ex}

\hspace{.8\funcindent}\begin{boxedminipage}{\funcwidth}

    \raggedright \textbf{evaluate\_eigenvalues\_at}(\textit{self}, \textit{nodes}, \textit{component}={\tt None}, \textit{as\_matrix}={\tt False})

    \vspace{-1.5ex}

    \rule{\textwidth}{0.5\fboxrule}
\setlength{\parskip}{2ex}
    Evaluate the eigenvalues
    $\lambda_i\ofs{x}$ at some grid
    nodes $\gamma$.

\setlength{\parskip}{1ex}
      \textbf{Parameters}
      \vspace{-1ex}

      \begin{quote}
        \begin{Ventry}{xxxxxxxxx}

          \item[nodes]

          The grid nodes $\gamma$ we want to evaluate the
          eigenvalues at.

          \item[component]

          The index $i$ of the eigenvalue $\lambda_i$
          that gets evaluated.

          \item[as\_matrix]

          Returns the whole matrix $\Lambda$ instead of
          only a list with the eigenvalues $\lambda_i$.

        \end{Ventry}

      \end{quote}

      \textbf{Return Value}
    \vspace{-1ex}

      \begin{quote}
      A sorted list with $2$ entries for the two eigenvalues evaluated at
      the nodes. Or a single value if a component was specified.

      \end{quote}



      Overrides: MatrixPotential.evaluate\_eigenvalues\_at

    \end{boxedminipage}

    \vspace{0.5ex}

\hspace{.8\funcindent}\begin{boxedminipage}{\funcwidth}

    \raggedright \textbf{calculate\_eigenvectors}(\textit{self})

    \vspace{-1.5ex}

    \rule{\textwidth}{0.5\fboxrule}
\setlength{\parskip}{2ex}
    Calculate the two eigenvectors $\nu_i\ofs{x}$ of the
    potential $V\ofs{x}$. We can do this by symbolical
    calculations.

\setlength{\parskip}{1ex}

\textbf{Note:} The eigenvectors are memoized for later reuse.

      Overrides: MatrixPotential.calculate\_eigenvectors




    \end{boxedminipage}

    \vspace{0.5ex}

\hspace{.8\funcindent}\begin{boxedminipage}{\funcwidth}

    \raggedright \textbf{evaluate\_eigenvectors\_at}(\textit{self}, \textit{nodes})

    \vspace{-1.5ex}

    \rule{\textwidth}{0.5\fboxrule}
\setlength{\parskip}{2ex}
    Evaluate the eigenvectors $\nu_i\ofs{x}$ at some
    grid nodes $\gamma$.

\setlength{\parskip}{1ex}
      \textbf{Parameters}
      \vspace{-1ex}

      \begin{quote}
        \begin{Ventry}{xxxxx}

          \item[nodes]

          The grid nodes $\gamma$ we want to evaluate the
          eigenvectors at.

        \end{Ventry}

      \end{quote}

      \textbf{Return Value}
    \vspace{-1ex}

      \begin{quote}
      A list with the two eigenvectors evaluated at the given nodes.

      \end{quote}



      Overrides: MatrixPotential.evaluate\_eigenvectors\_at

    \end{boxedminipage}

    \vspace{0.5ex}

\hspace{.8\funcindent}\begin{boxedminipage}{\funcwidth}

    \raggedright \textbf{project\_to\_eigen}(\textit{self}, \textit{nodes}, \textit{values}, \textit{basis}={\tt None})

    \vspace{-1.5ex}

    \rule{\textwidth}{0.5\fboxrule}
\setlength{\parskip}{2ex}
    Project a given vector from the canonical basis to the eigenbasis of
    the potential.

\setlength{\parskip}{1ex}
      \textbf{Parameters}
      \vspace{-1ex}

      \begin{quote}
        \begin{Ventry}{xxxxxx}

          \item[nodes]

          The grid nodes $\gamma$ for the pointwise
          transformation.

          \item[values]

          The list of vectors $\varphi_i$ containing the
          values we want to transform.

          \item[basis]

          A list of basisvectors $\nu_i$. Allows to use this function for
          external data, similar to a static function.

        \end{Ventry}

      \end{quote}

      \textbf{Return Value}
    \vspace{-1ex}

      \begin{quote}
      Returned is another list containing the projection of the values into
      the eigenbasis.

      \end{quote}



      Overrides: MatrixPotential.project\_to\_eigen

    \end{boxedminipage}

    \vspace{0.5ex}

\hspace{.8\funcindent}\begin{boxedminipage}{\funcwidth}

    \raggedright \textbf{project\_to\_canonical}(\textit{self}, \textit{nodes}, \textit{values}, \textit{basis}={\tt None})

    \vspace{-1.5ex}

    \rule{\textwidth}{0.5\fboxrule}
\setlength{\parskip}{2ex}
    Project a given vector from the potential's eigenbasis to the canonical
    basis.

\setlength{\parskip}{1ex}
      \textbf{Parameters}
      \vspace{-1ex}

      \begin{quote}
        \begin{Ventry}{xxxxxx}

          \item[nodes]

          The grid nodes $\gamma$ for the pointwise
          transformation.

          \item[values]

          The list of vectors $\varphi_i$ containing the values we want
          to transform.

          \item[basis]

          A list of basis vectors $\nu_i$. Allows to use this function
          for external data, similar to a static function.

        \end{Ventry}

      \end{quote}

      \textbf{Return Value}
    \vspace{-1ex}

      \begin{quote}
      Returned is another list containing the projection of the values into
      the eigenbasis.

      \end{quote}



      Overrides: MatrixPotential.project\_to\_canonical

    \end{boxedminipage}

    \vspace{0.5ex}

\hspace{.8\funcindent}\begin{boxedminipage}{\funcwidth}

    \raggedright \textbf{calculate\_exponential}(\textit{self}, \textit{factor}={\tt 1})

    \vspace{-1.5ex}

    \rule{\textwidth}{0.5\fboxrule}
\setlength{\parskip}{2ex}
    Calculate the matrix exponential $E =
    \exp\ofs{\alpha M}$. In this case the
    matrix is of size $2 \times 2$ thus the general exponential can
    be calculated analytically.

\setlength{\parskip}{1ex}
      \textbf{Parameters}
      \vspace{-1ex}

      \begin{quote}
        \begin{Ventry}{xxxxxx}

          \item[factor]

          A prefactor $\alpha$ in the exponential.

        \end{Ventry}

      \end{quote}



      Overrides: MatrixPotential.calculate\_exponential

    \end{boxedminipage}

    \vspace{0.5ex}

\hspace{.8\funcindent}\begin{boxedminipage}{\funcwidth}

    \raggedright \textbf{evaluate\_exponential\_at}(\textit{self}, \textit{nodes})

    \vspace{-1.5ex}

    \rule{\textwidth}{0.5\fboxrule}
\setlength{\parskip}{2ex}
    Evaluate the exponential of the potential matrix $V$ at some grid
    nodes $\gamma$.

\setlength{\parskip}{1ex}
      \textbf{Parameters}
      \vspace{-1ex}

      \begin{quote}
        \begin{Ventry}{xxxxx}

          \item[nodes]

          The grid nodes $\gamma$ we want to evaluate the
          exponential at.

        \end{Ventry}

      \end{quote}

      \textbf{Return Value}
    \vspace{-1ex}

      \begin{quote}
      The numerical approximation of the matrix exponential at the given
      grid nodes.

      \end{quote}



      Overrides: MatrixPotential.evaluate\_exponential\_at

    \end{boxedminipage}

    \vspace{0.5ex}

\hspace{.8\funcindent}\begin{boxedminipage}{\funcwidth}

    \raggedright \textbf{calculate\_jacobian}(\textit{self})

    \vspace{-1.5ex}

    \rule{\textwidth}{0.5\fboxrule}
\setlength{\parskip}{2ex}
    Calculate the Jacobian matrix for each component $V_{i,j}$ of the
    potential. For potentials which depend only on one variable $x$, this
    equals the first derivative.

\setlength{\parskip}{1ex}


      Overrides: MatrixPotential.calculate\_jacobian

    \end{boxedminipage}

    \vspace{0.5ex}

\hspace{.8\funcindent}\begin{boxedminipage}{\funcwidth}

    \raggedright \textbf{evaluate\_jacobian\_at}(\textit{self}, \textit{nodes}, \textit{component}={\tt None})

    \vspace{-1.5ex}

    \rule{\textwidth}{0.5\fboxrule}
\setlength{\parskip}{2ex}
    Evaluate the Jacobian at some grid nodes $\gamma$ for
    each component $V_{i,j}$ of the potential.

\setlength{\parskip}{1ex}
      \textbf{Parameters}
      \vspace{-1ex}

      \begin{quote}
        \begin{Ventry}{xxxxxxxxx}

          \item[nodes]

          The grid nodes $\gamma$ the Jacobian gets
          evaluated at.

          \item[component]

          The index tuple $\left(i,j \right)$ that specifies
          the potential's entry of which the Jacobian is evaluated.
          (Defaults to \textit{None} to evaluate all)

        \end{Ventry}

      \end{quote}

      \textbf{Return Value}
    \vspace{-1ex}

      \begin{quote}
      Either a list or a single value depending on the optional parameters.

      \end{quote}



      Overrides: MatrixPotential.evaluate\_jacobian\_at

    \end{boxedminipage}

    \vspace{0.5ex}

\hspace{.8\funcindent}\begin{boxedminipage}{\funcwidth}

    \raggedright \textbf{calculate\_hessian}(\textit{self})

    \vspace{-1.5ex}

    \rule{\textwidth}{0.5\fboxrule}
\setlength{\parskip}{2ex}
    Calculate the Hessian matrix for each component $V_{i,j}$ of the
    potential. For potentials which depend only on one variable $x$, this
    equals the second derivative.

\setlength{\parskip}{1ex}


      Overrides: MatrixPotential.calculate\_hessian

    \end{boxedminipage}

    \vspace{0.5ex}

\hspace{.8\funcindent}\begin{boxedminipage}{\funcwidth}

    \raggedright \textbf{evaluate\_hessian\_at}(\textit{self}, \textit{nodes}, \textit{component}={\tt None})

    \vspace{-1.5ex}

    \rule{\textwidth}{0.5\fboxrule}
\setlength{\parskip}{2ex}
    Evaluate the Hessian at some grid nodes $\gamma$ for
    each component $V_{i,j}$ of the potential.

\setlength{\parskip}{1ex}
      \textbf{Parameters}
      \vspace{-1ex}

      \begin{quote}
        \begin{Ventry}{xxxxxxxxx}

          \item[nodes]

          The grid nodes $\gamma$ the Hessian gets
          evaluated at.

          \item[component]

          The index tuple $\left(i,j \right)$ that specifies
          the potential's entry of which the Hessian is evaluated. (Or
          \textit{None} to evaluate all)

        \end{Ventry}

      \end{quote}

      \textbf{Return Value}
    \vspace{-1ex}

      \begin{quote}
      Either a list or a single value depending on the optional parameters.

      \end{quote}



      Overrides: MatrixPotential.evaluate\_hessian\_at

    \end{boxedminipage}

    \vspace{0.5ex}

\hspace{.8\funcindent}\begin{boxedminipage}{\funcwidth}

    \raggedright \textbf{calculate\_local\_quadratic}(\textit{self}, \textit{diagonal\_component})

    \vspace{-1.5ex}

    \rule{\textwidth}{0.5\fboxrule}
\setlength{\parskip}{2ex}
    Calculate the local quadratic approximation matrix $U$ of the
    potential's eigenvalues in $\Lambda$. This function is
    used for the homogeneous case and takes into account the leading
    component $\chi$.

\setlength{\parskip}{1ex}
      \textbf{Parameters}
      \vspace{-1ex}

      \begin{quote}
        \begin{Ventry}{xxxxxxxxxxxxxxxxxx}

          \item[diagonal\_component]

          Specifies the index $i$ of the eigenvalue
          $\lambda_i$ that gets expanded into a Taylor
          series $u_i$.

        \end{Ventry}

      \end{quote}



      Overrides: MatrixPotential.calculate\_local\_quadratic

    \end{boxedminipage}

    \vspace{0.5ex}

\hspace{.8\funcindent}\begin{boxedminipage}{\funcwidth}

    \raggedright \textbf{evaluate\_local\_quadratic\_at}(\textit{self}, \textit{nodes})

    \vspace{-1.5ex}

    \rule{\textwidth}{0.5\fboxrule}
\setlength{\parskip}{2ex}
    Numerically evaluate the local quadratic approximation matrix $U$ of
    the potential's eigenvalues in $\Lambda$ at the given
    grid nodes $\gamma$. This function is used for the
    homogeneous case and takes into account the leading component
    $\chi$.

\setlength{\parskip}{1ex}
      \textbf{Parameters}
      \vspace{-1ex}

      \begin{quote}
        \begin{Ventry}{xxxxx}

          \item[nodes]

          The grid nodes $\gamma$ we want to evaluate the
          quadratic approximation at.

        \end{Ventry}

      \end{quote}

      \textbf{Return Value}
    \vspace{-1ex}

      \begin{quote}
      A list of arrays containing the values of $U_{i,j}$ at the nodes
      $\gamma$.

      \end{quote}



      Overrides: MatrixPotential.evaluate\_local\_quadratic\_at

    \end{boxedminipage}

    \vspace{0.5ex}

\hspace{.8\funcindent}\begin{boxedminipage}{\funcwidth}

    \raggedright \textbf{calculate\_local\_remainder}(\textit{self}, \textit{diagonal\_component})

    \vspace{-1.5ex}

    \rule{\textwidth}{0.5\fboxrule}
\setlength{\parskip}{2ex}
    Calculate the non-quadratic remainder matrix $W$ of the quadratic
    approximation matrix $U$ of the potential's eigenvalue matrix
    $\Lambda$. This function is used for the homogeneous
    case and takes into account the leading component
    $\chi$.

\setlength{\parskip}{1ex}
      \textbf{Parameters}
      \vspace{-1ex}

      \begin{quote}
        \begin{Ventry}{xxxxxxxxxxxxxxxxxx}

          \item[diagonal\_component]

          Specifies the index $\chi$ of the leading
          component $\lambda_\chi$.

        \end{Ventry}

      \end{quote}



      Overrides: MatrixPotential.calculate\_local\_remainder

    \end{boxedminipage}

    \vspace{0.5ex}

\hspace{.8\funcindent}\begin{boxedminipage}{\funcwidth}

    \raggedright \textbf{evaluate\_local\_remainder\_at}(\textit{self}, \textit{position}, \textit{nodes}, \textit{component}={\tt None})

    \vspace{-1.5ex}

    \rule{\textwidth}{0.5\fboxrule}
\setlength{\parskip}{2ex}
    Numerically evaluate the non-quadratic remainder matrix $W$ of the
    quadratic approximation matrix $U$ of the potential's eigenvalues in
    $\Lambda$ at the given nodes
    $\gamma$. This function is used for the homogeneous
    and the inhomogeneous case and just evaluates the remainder matrix
    $W$.

\setlength{\parskip}{1ex}
      \textbf{Parameters}
      \vspace{-1ex}

      \begin{quote}
        \begin{Ventry}{xxxxxxxxx}

          \item[position]

          The point $q$ where the Taylor series is computed.

          \item[nodes]

          The grid nodes $\gamma$ we want to evaluate the
          potential at.

          \item[component]

          The component $\left(i,j \right)$ of the remainder
          matrix $W$ that is evaluated.

        \end{Ventry}

      \end{quote}

      \textbf{Return Value}
    \vspace{-1ex}

      \begin{quote}
      A list with a single entry consisting of an array containing the
      values of $W$ at the nodes $\gamma$.

      \end{quote}



      Overrides: MatrixPotential.evaluate\_local\_remainder\_at

    \end{boxedminipage}

    \vspace{0.5ex}

\hspace{.8\funcindent}\begin{boxedminipage}{\funcwidth}

    \raggedright \textbf{calculate\_local\_quadratic\_multi}(\textit{self})

    \vspace{-1.5ex}

    \rule{\textwidth}{0.5\fboxrule}
\setlength{\parskip}{2ex}
    Calculate the local quadratic approximation matrix $U$ of all the
    potential's eigenvalues in $\Lambda$. This function is
    used for the inhomogeneous case.

\setlength{\parskip}{1ex}


      Overrides: MatrixPotential.calculate\_local\_quadratic\_multi

    \end{boxedminipage}

    \vspace{0.5ex}

\hspace{.8\funcindent}\begin{boxedminipage}{\funcwidth}

    \raggedright \textbf{evaluate\_local\_quadratic\_multi\_at}(\textit{self}, \textit{nodes}, \textit{component}={\tt None})

    \vspace{-1.5ex}

    \rule{\textwidth}{0.5\fboxrule}
\setlength{\parskip}{2ex}
    Numerically evaluate the local quadratic approximation matrix $U$ of
    the potential's eigenvalues in $\Lambda$ at the given
    grid nodes $\gamma$. This function is used for the
    inhomogeneous case.

\setlength{\parskip}{1ex}
      \textbf{Parameters}
      \vspace{-1ex}

      \begin{quote}
        \begin{Ventry}{xxxxxxxxx}

          \item[nodes]

          The grid nodes $\gamma$ we want to evaluate the
          quadratic approximation at.

          \item[component]

          The component $\left(i,j \right)$ of the quadratic
          approximation matrix $U$ that is evaluated.

        \end{Ventry}

      \end{quote}

      \textbf{Return Value}
    \vspace{-1ex}

      \begin{quote}
      A list of arrays or a single array containing the values of
      $U_{i,j}$ at the nodes $\gamma$.

      \end{quote}



      Overrides: MatrixPotential.evaluate\_local\_quadratic\_multi\_at

    \end{boxedminipage}

    \vspace{0.5ex}

\hspace{.8\funcindent}\begin{boxedminipage}{\funcwidth}

    \raggedright \textbf{calculate\_local\_remainder\_multi}(\textit{self})

    \vspace{-1.5ex}

    \rule{\textwidth}{0.5\fboxrule}
\setlength{\parskip}{2ex}
    Calculate the non-quadratic remainder matrix $W$ of the quadratic
    approximation matrix $U$ of the potential's eigenvalue matrix
    $\Lambda$. This function is used for the inhomogeneous
    case.

\setlength{\parskip}{1ex}


      Overrides: MatrixPotential.calculate\_local\_remainder\_multi

    \end{boxedminipage}

\newpage

%%%%%%%%%%%%%%%%%%%%%%%%%%%%%%%%%%%%%%%%%%%%%%%%%%%%%%%%%%%%%%%%%%%%%%%%%%%
%%                          Instance Variables                           %%
%%%%%%%%%%%%%%%%%%%%%%%%%%%%%%%%%%%%%%%%%%%%%%%%%%%%%%%%%%%%%%%%%%%%%%%%%%%

  \subsection{Instance Variables}

    \vspace{-1cm}
\hspace{\varindent}\begin{longtable}{|p{\varnamewidth}|p{\vardescrwidth}|l}
\cline{1-2}
\cline{1-2} \centering \textbf{Name} & \centering \textbf{Description}& \\
\cline{1-2}
\endhead\cline{1-2}\multicolumn{3}{r}{\small\textit{continued on next page}}\\\endfoot\cline{1-2}
\endlastfoot\raggedright x\- & The variable $x$ that represents position space.&\\
\cline{1-2}
\raggedright p\-o\-t\-e\-n\-t\-i\-a\-l\- & The matrix of the potential $V\ofs{x}$.&\\
\cline{1-2}
\end{longtable}

    \index{MatrixPotential2S \textit{(module)}!MatrixPotential2S \textit{(class)}|)}
