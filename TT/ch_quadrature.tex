\begin{chapter}{Inner products, integrals and quadrature}
\label{ch:quadrature}

\section{A hierarchy of brakets}

In this chapter we will develop and summarize all necessary tools related to
inner products of wavepackets. We will follow a top down approach and start with
the braket of a full vector valued wavepacket $\Ket{\Psi}$ which may be homogeneous
or inhomogeneous at the moment. The primes indicate that the bra and the ket can
have different parameter sets $\Pi$. This is obvious when $\Ket{\Psi}$ is an
inhomogeneous wavepacket. Also we include an operator $F$ which may be the identity.

\begin{equation}
\begin{split}
  \Braket{\Psi | F | \Psi^\prime} & =
    \Bra{
    \begin{pmatrix}
    \Phi_0     \\
    \vdots     \\
    \Phi_{N-1}
    \end{pmatrix}
    }
    \begin{pmatrix}
    {}     & \vdots  & {} \\
    \hdots & F_{r,c} & {} \\
    {}     &         & {}
    \end{pmatrix}
    \Ket{
    \begin{pmatrix}
    \Phi_0^\prime     \\
    \vdots     \\
    \Phi_{N-1}^\prime
    \end{pmatrix}
    } \\
  & = \sum_{r=0}^{N-1} \sum_{c=0}^{N-1} \Braket{\Phi_r | F_{r,c} | \Phi_c^\prime}
\end{split}
\end{equation}

where $F$ is a $N \times N$ block matrix consisting of $K \times K$ blocks
denoted by $F_{i,j} \rassign f$. We then consider a single term out of this
double sum

\begin{equation}
\begin{split}
  \Braket{\Phi_i | f | \Phi_j^\prime} & =
    \Bra{
    \begin{pmatrix}
    \phi_0     \\
    \vdots     \\
    \phi_{K-1}
    \end{pmatrix}
    }
    \begin{pmatrix}
    {}     & \vdots  & {} \\
    \hdots & f_{k,l} & {} \\
    {}     &         & {}
    \end{pmatrix}
    \Ket{
    \begin{pmatrix}
    \phi_0^\prime     \\
    \vdots     \\
    \phi_{K-1}^\prime
    \end{pmatrix}
    } \\
  & = \sum_{k=0}^{K-1} \sum_{l=0}^{K-1} \Braket{\phi_k | f_{k,l} | \phi_l^\prime}
\end{split}
\end{equation}

where $\phi_i$ are the basis functions from \eqref{eq:hagedorn_kstate_1d}. Now we
pick again a single entry out of the double sum only and find the following
integral at the bottom of this hierarchy

\begin{equation} \label{eq:single_entry_braket}
  \Braket{\phi_k | f_{k,l} | \phi_l^\prime} = \int f_{k,l}\ofs{x} \conj{\phi_k\ofs{x}} \phi_l^\prime\ofs{x} \,.
\end{equation}

We now want to find an efficient way to calculate this integral for all $i$, $j$, $k$ and $l$
or, in other words, set up the matrix $F$. This integral can almost never be solved
analytically thus the integral is approximated by high order quadrature.


\section{Inner products of basis functions}

We speak of inhomogeneous inner products if the part in the bra and the part in
the ket of \eqref{eq:single_entry_braket} have different Hagedorn parameters sets $\Pi$.
In this case, all formulae become much more complicated.

\subsection{An analytical ansatz to inner products}

The inner product of two basis functions which have different sets of
parameters denoted by $\Pi_k \assign \{P_k, Q_k, S_k, p_k, q_k\}$ and
$\Pi_l \assign \{P_l, Q_l, S_l, p_l, q_l\}$ respectively is written as usual as
$\Braket{ \phi^k | \phi^l }$. This is the expression we want to evaluate now and
we can even write down a closed form solution based on induction and the recursion
relation for Hermite polynomials. The expression for the ground states $\phi_0$
acts as induction base and is given by

\begin{multline} \label{eq:multifamily_innerp_ground}
  \Braket{ \phi_0^k | \phi_0^l } =
  \sqrt{\frac{-2 i}{Q_2 \conj{P_1} - P_2 \conj{Q_1}}} \cdot
    \exp \Biggl(
      \frac{i}{2 \varepsilon^2}
      \frac{Q_2 \conj{Q_1} \left(p_2-p_1\right)^2 + P_2 \conj{P_1} \left(q_2-q_1\right)^2}
            {\left(Q_2 \conj{P_1} - P_2 \conj{Q_1}\right)}
    \\
    -\frac{i}{\varepsilon^2}
    \frac{\left(q_2-q_1\right) \left( Q_2 \conj{P_1} p_2 - P_2 \conj{Q_1} p_1\right)}
         {\left(Q_2 \conj{P_1} - P_2 \conj{Q_1}\right)}
    \Biggr) \,.
\end{multline}

For the inner product of higher level functions $\phi_i$ the whole thing gets much
more complicated

\begin{multline} \label{eq:mixing_braket}
  \Braket{ \phi_k^k | \phi_l^l } =
  \frac{1}{\sqrt{k!l!}} 2^{-\frac{k+l}{2}} \Braket{ \phi_0^k | \phi_0^l } \cdot
  \left(i \conj{ P_1} Q_2 - i \conj{Q_1} P_2\right)^{-\frac{k+l}{2}} \cdot \\
  \sum_{j=0}^{\min\left(k,l\right)}
    \Biggl(
      \binom{k}{j} \binom{l}{j} j! 4^j
      \left(i Q_2  P_1 - i Q_1  P_2\right)^{\frac{k-j}{2}}
      \left(i \conj{Q_2 P_1} - i\conj{Q_1 P_2}\right)^{\frac{l-j}{2}}
      \\
      \cdot H_{k-j}\left(-\frac{1}{\varepsilon}
                    \frac{Q_2\left(p_1-p_2\right)-P_2\left(q_1-q_2\right)}
                         {\sqrt{Q_2 P_1 - Q_1 P_2}\sqrt{\conj{P_1}Q_2-\conj{Q_1} P_2}}\right)
      \\
      \cdot H_{l-j}\left(\frac{1}{\varepsilon}
                   \frac{\conj{ P_1}\left(q_1-q_2\right)-\conj{Q_1}\left(p_1-p_2\right)}
                        {\sqrt{\conj{Q_2 P_1}-\conj{Q_1 P_2}}\sqrt{\conj{ P_1}Q_2-\conj{Q_1} P_2}}\right)
    \Biggr) \,.
\end{multline}

For the proofs of these formulae see reference \cite{H_R_quantization_rules}.

Despite we can evaluate the inner product and have a closed form solution for arbitrary
wavefunctions, these formulae are unsuitable for numerical calculation. There are
several reasons but for example the factorials and binomial coefficients lead to
overflow even for relatively small $k$ and $l$. Further the sum may be numerically
unstable. Thus we need to find a better way to perform these calculations.


\subsection{Inhomogeneous or mixing quadrature rule}
\label{sec:mixing_quadrature}

In this section we will develop a quadrature rule to evaluate the brakets in \eqref{eq:mixing_braket}.
First we notice that each $\phi$ which is given by \eqref{eq:hagedorn_kstate_1d}
is represented through a mathematical expression of the general form

\begin{equation} \label{eq:expression_format}
  C \cdot P^n\ofs{\xi} \cdot \exp \ofs{\theta}
\end{equation}

consisting of an arbitrary constant $C \in \mathbb{C}$, a polynomial $P^n\ofs{\cdot}$
of degree $n$ and an exponential $\exp\ofs{\cdot}$. We try a new ansatz for calculating
the inner product. Evaluating the braket $\Braket{\phi^k | \phi^l}$ results in a
multiplication of two expressions of the form \eqref{eq:expression_format}. The
parts in this expression can be grouped by same type

\begin{equation} \label{eq:expression_reformat}
\begin{split}
  \Braket{\phi^k | \phi^l} & =
  \int_\mathbb{R} \conj{C_k P^n_k\ofs{\xi_k} \exp \ofs{\theta_k}} C_l P^n_l\ofs{\xi_l} \exp \ofs{\theta_l} \,dx \\
  & =   \int_\mathbb{R} \conj{C_k} C_l P^n_k\ofs{\conj{\xi_k}} P^n_l\ofs{\xi_l} \exp \ofs{\conj{\theta_k}} \exp \ofs{\theta_l} \,dx \,.
\end{split}
\end{equation}

Let's take a closer look at the integrand of this expression now. With Gauss Hermite
quadrature in mind we are especially interested in the exponential parts. They have
a general form like

\begin{equation} \label{eq:expression_format_exponential}
  \exp\ofs{\theta} = \exp\left(s \cdot \left(x-m\right)^2 + \cdots \right) \,.
\end{equation}

We concentrate on the exponentials of \eqref{eq:expression_reformat} only. We combine
them and distribute the complex conjugate onto the variables affected

\begin{equation*}
\begin{split}
    & \conj{\exp\left(\frac{i}{2\varepsilon^2} P_k Q_k^{-1} \left(x-q_k\right)^2 + \frac{i}{\varepsilon^2} p_k \left( x-q_k \right)\right)}
      \cdot \exp\left(\frac{i}{2\varepsilon^2} P_l Q_l^{-1} \left(x-q_l\right)^2 + \frac{i}{\varepsilon^2} p_l \left( x-q_l \right)\right) \\
  = & \exp\left(\conj{\frac{i}{2\varepsilon^2} P_k Q_k^{-1} \left(x-q_k\right)^2 + \frac{i}{\varepsilon^2} p_k \left( x-q_k \right)}
              + \frac{i}{2\varepsilon^2} P_l Q_l^{-1} \left(x-q_l\right)^2 + \frac{i}{\varepsilon^2} p_l \left( x-q_l \right) \right) \\
  = & \exp\left( -\frac{i}{2\varepsilon^2} \conj{P_k} \conj{Q_k^{-1}} \left(x-q_k\right)^2 - \frac{i}{\varepsilon^2} p_k \left( x-q_k \right)
              + \frac{i}{2\varepsilon^2} P_l Q_l^{-1} \left(x-q_l\right)^2 + \frac{i}{\varepsilon^2} p_l \left( x-q_l \right) \right) \,.
\end{split}
\end{equation*}

For the sake of readability we define the following variables

\begin{equation}
\begin{split}
  r_k & \assign P_k Q_k^{-1} \\
  r_l & \assign P_l Q_l^{-1} \,.
\end{split}
\end{equation}

Plugging these into the equation above and expanding the squares we get for the
exponent

\begin{equation*}
  \frac{i}{\varepsilon^2} \left( - \frac{1}{2}\underbrace{\left(\conj{r_k}-r_l\right)}_{\alpha + i \beta} x^2
                                 + \underbrace{\left(\conj{r_k}q_k-r_l q_l\right)}_{\gamma + i \delta} x
                                 - \underbrace{\frac{1}{2}\left(\conj{r_k}q_k^2 + r_l q_l^2\right)
                                 + \left(p_l-p_k\right) x
                                 + p_k q_k - p_l q_l}_{\text{junk}}
                          \right)
\end{equation*}

where we introduced two new complex variables $\alpha + i \beta$ and $\gamma + i \delta$
and sorted out some unimportant expressions. We now carry out the multiplication
with respect to the real parts of these complex numbers. This yields

\begin{equation}
  -\frac{1}{\varepsilon^2} \left( -\frac{\beta}{2}x^2 + i \left(\cdots\right)
                                  + \delta x -i \left(\cdots\right)
                                  + \cdots
                          \right)
\end{equation}

where the dots indicate more junk. To get back to a form along the lines of
\eqref{eq:expression_format_exponential} we have to complete the square, first
divide out a factor of $-\frac{\beta}{2}$ and then complete by a factor of
$\left(\frac{\delta}{\beta}\right)^2$

\begin{equation}
\begin{split}
    & -\frac{1}{\varepsilon^2} \left( x^2 -\frac{2\delta}{\beta} x + \ldots \right) \left(-\frac{\beta}{2}\right) \\
%   = & -\frac{1}{\varepsilon^2} \left( x^2 -\frac{2\delta}{\beta} x + \left(\frac{\delta}{\beta}\right)^2 - \left(\frac{\delta}{\beta}\right)^2 \right) \left(-\frac{\beta}{2}\right) \\
%   = & -\frac{1}{\varepsilon^2} \left( \left(-\frac{\beta}{2}\right) \left(x-\frac{\delta}{\beta}\right)^2 - \left(-\frac{\beta}{2}\right)\left(\frac{\delta}{\beta}\right)^2 \right) \\
  = & -\frac{1}{\varepsilon^2} \left( \left(-\frac{\beta}{2}\right) \left(x-\frac{\delta}{\beta}\right)^2 + \frac{\delta^2}{2\beta} \right) \,.
\end{split}
\end{equation}

From this last expression we get the quadrature rule components $s \assign Q_0$
and $m \assign q_0$ of expression \eqref{eq:expression_format_exponential} as

\begin{equation} \label{eq:inhomog_quad_params}
\begin{split}
  q_0 & \assign \frac{\delta}{\beta} = \frac{\Im \left(\conj{r_k}q_k-r_l q_l\right)}{\Im \left(\conj{r_k}-r_l\right)} \\
  Q_0 & \assign -\frac{\beta}{2} = -\frac{\Im \left(\conj{r_k}-r_l\right)}{2} \\
  Q_S & \assign \frac{1}{\sqrt{Q_0}} \,.
\end{split}
\end{equation}

Now we transform the nodes $\gamma_i$ according to the weighted position mean $q_0$
and the parameter $Q_S$ which changes the spread of the nodes. This yields

\begin{equation} \label{eq:inhomog_transformed_nodes}
  \gamma^{\prime}_i \assign q_0 + \varepsilon \cdot Q_S \cdot \gamma_i
\end{equation}

for the mixing quadrature nodes which are located in the space around where
the product of $\phi_k$ and $\phi_l$ is maximal. A procedure that calculates
$q_0$, $Q_S$ and the adapted quadrature nodes is given by algorithm \ref{al:mixing_hagedorn_parameters}.

\begin{algorithm}
\caption{Mixing two sets $\Pi_r$ and $\Pi_c$ of Hagedorn parameters}
\label{al:mixing_hagedorn_parameters}
\begin{algorithmic}
  \REQUIRE Two sets $\Pi_r$ and $\Pi_c$ of Hagedorn parameters
  \REQUIRE A quadrature rule $\left(\gamma_i, \omega_i\right)$
  \STATE // Apply the mixing formula to the parameters
  \STATE $r_r \assign \frac{P_r}{Q_r}$
  \STATE $r_c \assign \frac{P_c}{Q_c}$

  \STATE $q_0 \assign \frac{\Im \left(\conj{r_r}q_r-r_c q_c\right)}{\Im \left(\conj{r_r}-r_c\right)}$
  \STATE $Q_0 \assign -\frac{\Im \left(\conj{r_r}-r_c\right)}{2}$
  \STATE $Q_S \assign \frac{1}{\sqrt{Q_0}}$

  \STATE // And shift the quadrature nodes
  \STATE $\gamma^\prime \assign q_0 + \varepsilon Q_S \gamma$
  \RETURN $q_0$ and $Q_S$ and $\gamma^\prime$
\end{algorithmic}
\end{algorithm}

We get back to the homogeneous case if we choose the sets $\Pi_k$ and
$\Pi_l$ of Hagedorn parameters identical.

Note that we get issues if at any time it happens that

\begin{equation}
  \Im \left(\frac{\conj{P_k}}{\conj{Q_k}} -\frac{P_l}{Q_l}\right) > 0 \,.
\end{equation}


\subsection{Homogeneous quadrature rule}

In this section we reduce the mixing quadrature rule of the last section to the
homogeneous case where everything becomes much simpler.
We will start from the assumption that the mixing quadrature rule contains the
homogeneous one as a special case. (This is not just speculation but can be shown
by direct calculation analogous to what we did in the last section.) Suppose for
this section that both $\phi_i$ in \eqref{eq:single_entry_braket} belong to the
same basis and have an identical parameter set $\Pi$ hence $\Pi_k = \Pi_l$ or at
least $P_k = P_l$, $Q_k = Q_l$ and $p_k = p_l$, $q_k = q_l$. With this assumption
we simplify the quadrature formulae \eqref{eq:inhomog_quad_params} and
\eqref{eq:inhomog_transformed_nodes} to the homogeneous case.

Let's start with simplification of the $Q_0$ parameter

\begin{equation}
\begin{split}
  Q_0 & \assign -\frac{\Im \left(\conj{r_k}-r_l\right)}{2}
        = -\frac{1}{2} \Im \left(\conj{r_k}-r_k\right) \\
      & = - \frac{1}{2} \left( - \frac{2}{\conj{Q}Q}\right) \\
      & = \frac{1}{|Q|^2} \,.
\end{split}
\end{equation}

The parameter $Q_S$ is then as trivial as

\begin{equation}
  Q_S \assign \frac{1}{\sqrt{Q_0}} = \frac{1}{\sqrt{\frac{1}{|Q|^2}}} = |Q| \,.
\end{equation}

In the last case we have for the weighted position

\begin{equation}
\begin{split}
  q_0 & \assign \frac{\Im \left(\conj{r_k}q_k-r_l q_l\right)}{\Im \left(\conj{r_k}-r_l\right)}
        = \frac{\Im \left(\conj{r_k}q_k-r_k q_k\right)}{\Im \left(\conj{r_k}-r_k\right)} \\
%       & = \frac{\Im \left(\left(\conj{r_k}-r_k\right) q\right)}{-\frac{2}{\conj{Q}Q}}
      & = \frac{q \Im \left(\conj{r_k}-r_k\right)}{-\frac{2}{\conj{Q}Q}}
        = \frac{-\frac{2}{\conj{Q}Q}q}{-\frac{2}{\conj{Q}Q}} \\
      & = q
\end{split}
\end{equation}

which should be clear also without computation. For the transformed quadrature
nodes we can write

\begin{equation}
  \gamma^{\prime}_i \assign q + \varepsilon \cdot |Q| \cdot \gamma_i
\end{equation}

which was shown to work in real simulations for homogeneous wavepackets.


% \subsubsection{Direct computation}
%
% The above result is consistent with what we get directly from the following computations
% starting with the already simplified exponentials. Just for comparison here is the
% direct derivation of the homogeneous quadrature rule.
%
% \begin{align} \label{eq:expression_exponentials_combined}
%   & \conj{\exp\left(\frac{i}{2\varepsilon^2} P Q^{-1} \left(x-q\right)^2 + \frac{i}{\varepsilon^2} p \left( x-q \right)\right)}
%   \cdot \exp\left(\frac{i}{2\varepsilon^2} P Q^{-1} \left(x-q\right)^2 + \frac{i}{\varepsilon^2} p \left( x-q \right)\right) \nonumber \\
%   = & \exp\left(\conj{\frac{i}{2\varepsilon^2} P Q^{-1} \left(x-q\right)^2 + \frac{i}{\varepsilon^2} p \left( x-q \right)}
%               + \frac{i}{2\varepsilon^2} P Q^{-1} \left(x-q\right)^2 + \frac{i}{\varepsilon^2} p \left( x-q \right) \right) \nonumber \\
%   = & \exp\left( -\frac{i}{2\varepsilon^2} \conj{P} \conj{Q^{-1}} \left(x-q\right)^2 - \frac{i}{\varepsilon^2} p \left( x-q \right)
%               + \frac{i}{2\varepsilon^2} P Q^{-1} \left(x-q\right)^2 + \frac{i}{\varepsilon^2} p \left( x-q \right) \right) \,.
% \end{align}
%
% Now we group the terms with an $x^2$ but without expanding the binomals. Thus we get
%
% \begin{align}
%     & -\frac{i}{2\varepsilon^2} \conj{P} \conj{Q^{-1}} \left(x-q\right)^2 + \frac{i}{2\varepsilon^2} P Q^{-1} \left(x-q\right)^2 - \frac{i}{\varepsilon^2} p \left( x-q \right)
%                + \frac{i}{\varepsilon^2} p \left( x-q \right) \\
%   = & -\frac{i}{2\varepsilon^2} \conj{P} \conj{Q^{-1}} \left(x-q\right)^2 + \frac{i}{2\varepsilon^2} P Q^{-1} \left(x-q\right)^2 + \bigO{x} \\
%   = & \frac{i}{\varepsilon^2} \left( -\frac{1}{2} \conj{P} \conj{Q^{-1}} \left(x-q\right)^2 + \frac{1}{2} P Q^{-1} \left(x-q\right)^2 \right) + \ldots \\
%   = & \frac{i}{\varepsilon^2} \left( \left(-\frac{1}{2} \conj{P} \conj{Q^{-1}} + \frac{1}{2} P Q^{-1} \right) \left(x-q\right)^2 \right) + \ldots \\
%   = & \frac{i}{\varepsilon^2} \left( -\frac{1}{2} \underbrace{\left( \frac{\conj{P}}{\conj{Q}} - \frac{P}{Q} \right)}_{\text{*}} \left(x-q\right)^2 \right) + \ldots \\
%   = & \frac{i}{\varepsilon^2} \left( -\frac{1}{2} \left( -2i \frac{1}{\conj{Q}Q} \right) \left(x-q\right)^2 \right) + \ldots \\
%   = & -\frac{1}{\varepsilon^2} \left( \frac{1}{\conj{Q}Q} \left(x-q\right)^2 \right) + \ldots \\
%   = & -\frac{1}{\varepsilon^2} \left( \underbrace{\frac{1}{|Q|^2}}_{Q_0} \left(x-\underbrace{q}_{q_0}\right)^2 \right) + \ldots \\
% \end{align}
%
% from which we again get for the quadrature nodes
%
% \begin{equation}
%   \gamma^{\prime}_i \assign q + \varepsilon \cdot |Q| \cdot \gamma_i
% \end{equation}


\section{Quadrature rules applied}

After we discussed in details the transformation of the quadrature nodes in the
last section we now look at the final quadrature rule. It's not really difficult,
but it is good to write down all the details at least once.

Assume we have the transformed quadrature nodes $\gamma_i^\prime$ given by \eqref{eq:inhomog_transformed_nodes}.
We now carry out the quadrature for resolving equation \eqref{eq:single_entry_braket}

\begin{equation}
\begin{split}
  \Braket{\phi_k | f | \phi_l^\prime}
  \approx
  \varepsilon \cdot Q_S \cdot \sum_{r=0}^R \conj{\phi_k\ofs{\gamma_r^\prime}} \cdot f\ofs{\gamma_r^\prime} \cdot \phi_l^\prime\ofs{\gamma_r^\prime} \cdot \omega_r
\end{split}
\end{equation}

where the two $\phi$ in general belong to the different families. But if they really
have the same parameter sets $\Pi$ then we can simplify this formula slightly. The
trick is that we omit a prefactor of $\frac{1}{\sqrt{Q}}$ when evaluating $\phi\ofs{\gamma_r^\prime}$.
Because we have two times this evaluation, both prefactors accumulate to $\frac{1}{|Q|}$.
In the homogeneous case we have $Q_S = |Q|$ hence the $Q_S$ outside the sum
cancels nicely with this prefactors. And the above formula becomes

\begin{equation} \label{eq:homogeneous_quadrature_rule}
\begin{split}
  \Braket{\phi_k | f | \phi_l}
  \approx
  \varepsilon \cdot \sum_{r=0}^R \conj{\phi_k\ofs{\gamma_r^\prime}} \cdot f\ofs{\gamma_r^\prime} \cdot \phi_l\ofs{\gamma_r^\prime} \cdot \omega_r \,.
\end{split}
\end{equation}

\end{chapter}
