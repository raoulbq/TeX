\chapter{Wavepacket Propagation}
\label{ch:wavepacket_propagation}


This last theoretical chapter is about time propagation of semi-classical
wavepackets. The last remaining missing piece of this puzzle is a good scheme
for time propagation of semi-classical Hagedorn wavepackets. Such an algorithm was
first described in \cite{FGL_semiclassical_dynamics} for the adiabatic one-level
case. The generalisation to multiple energy levels was done in \cite{B_bachelor_thesis}
but for one space dimension only. These results were published in \cite{BGH_nonadibatic_algorithms}.

In this chapter we review this algorithm once more and show a generalisation
to several energy levels in an arbitrary dimensional space.


\section{Time propagation of Hagedorn wavepackets}


The first step is to consider the free particle Schrödinger equation which
contains only the kinetic operator $T$ and $V \equiv 0$:

\begin{equation*}
  i \varepsilon^2  \pdiff{\Psi}{t} = -\frac{1}{2} \varepsilon^4 \Delta \Psi \,.
\end{equation*}

The proposition 2.1 of \cite{FGL_semiclassical_dynamics} tells us that a Hagedorn wavepacket
$\Phi[\Pi]$ of the form \eqref{eq:scalar_wavepacket} solves the free Schrödinger
equation with the following time evolution of its parameter set $\Pi(t)$:

\begin{equation} \label{eq:update_scheme_kin}
\begin{split}
  \vec{q}(t) & = \vec{q}(0) + t \, \mat{M}\inv \vec{p}(0) \\
  \mat{Q}(t) & = \mat{Q}(0) + t \, \mat{M}\inv \mat{P}(0) \\
  S(t) & = S(0) + \frac{1}{2}t \, \vec{p}(0)\T \mat{M}\inv \vec{p}(0)
\end{split}
\end{equation}

where $\vec{p}(t) = \vec{p}(0)$ and $\mat{P}(t) = \mat{P}(0)$ remain unchanged.
Also unchanged are the coefficients $\{c_{\vec{k}}\}_{\vec{k} \in \mathfrak{K}}$
of $\Phi$. The matrix $\mat{M}$ is the mass scaling matrix. It is set to the
identity if nothing else is written. For details see \cite[equation 2.8]{FGL_semiclassical_dynamics}.
A free particle is not that interesting. The important point is that we can
solve its quantum equation of motion by only evolving the parameters $\Pi(t)$
using the simple update scheme above.

Next we consider the potential equation lacking the kinetic part:

\begin{equation*}
  i \varepsilon^2  \pdiff{\Psi}{t} = U(\vec{x}) \Psi \,.
\end{equation*}

For a first approach we assume $U(\vec{x})$ to be quadratic at most. The
proposition 2.2 of \cite{FGL_semiclassical_dynamics} gives us another set of update rules
for $\Pi(t)$:

\begin{equation} \label{eq:update_scheme_pot}
\begin{split}
  \vec{p}(t) & = \vec{p}(0) - t \, \nabla U(\vec{q}(0)) \\
  \mat{P}(t) & = \mat{P}(0) - t \, \nabla^2 U(\vec{q}(0)) \mat{Q}(0) \\
  S(t) & = S(0) - t \, U(\vec{q}(0))
\end{split}
\end{equation}

and $\vec{q}(t) = \vec{q}(0)$ and $\mat{Q}(t) = \mat{Q}(0)$. In this case the
coefficients $\{c_{\vec{k}}\}_{\vec{k} \in \mathfrak{K}}$ are left alone. Using
the formulae from both sets \eqref{eq:update_scheme_kin} and \eqref{eq:update_scheme_pot}
together we can solve the time dependent harmonic oscillator. All we have to do is to
propagate in time the parameter set $\Pi(t)$ of $\Phi[\Pi]$. In the whole
process we never touch or change the coefficients $\{c_{\vec{k}}\}_{\vec{k} \in \mathfrak{K}}$.
This is the reason that makes this time propagation scheme extraordinarily
efficient. But this is not really astonishing given that the basis functions
$\phi_{\vec{k}}$ are just generalised eigenstates of the harmonic oscillator.
(For more details see the paper \cite{H_ladder_operators}).

Our next step aims for handling almost arbitrary potentials. But dropping the
restriction of $U$ being quadratic provides us with several new hurdles.
Assume the potential $V(\vec{x})$ is not quadratic anymore. We can always split
$V$ into a quadratic part $U$ and the non-quadratic remainder $W$:

\begin{equation*}
  V(\vec{x}) = U(\vec{x}) + W(\vec{x}) \,.
\end{equation*}

This is done by a simple Taylor expansion of $V(\vec{x})$ around a point
\footnote{It's not a coincidence that we call the expansion point $\vec{q}$ here.}
$\vec{q}$ giving:

\begin{align*}
  U(\vec{x}) & = V(\vec{q}) + \nabla V(\vec{q}) (\vec{x}-\vec{q})
                 + \frac{1}{2} (\vec{x}-\vec{q})\T \nabla^2 V(\vec{q}) (\vec{x}-\vec{q}) \\
  W(\vec{x}) & = V(\vec{x}) - U(\vec{x}) \,.
\end{align*}

This expansion allows us to focus on $W(\vec{x})$ solely while assuming that
$U(\vec{x})$ is perfectly handled by \eqref{eq:update_scheme_pot}. We look at
the potential equation:

\begin{equation*}
  i \varepsilon^2  \pdiff{\Psi}{t} = W(\vec{x}) \Psi
\end{equation*}

once more but this time using the non-quadratic remainder part $W(\vec{x})$.
Since $W$ can be arbitrarily complicated there is no hope to solve this
equation analytically by a simple scheme as we did before. The trick is to
perform a Galerkin approximation to solve this equation. To do this we need
a Hilbert space of test functions $v(\vec{x})$ defined by:

\begin{equation}
  \mathcal{M} \assign \left\{ v \in \mathrm{L}^2(\mathbb{R}^D) \colon
                              v(\vec{x}) \assign \sum_{\vec{k}\in\mathfrak{K}}
                                                 c_{\vec{k}} \phi_{\vec{k}}[\Pi](\vec{x})
                         \,, c_{\vec{k}} \in \mathbb{C}
                      \right\} \,.
\end{equation}

At this point we may repeat that $\phi_{\vec{k}}$ constitutes a (complete) basis
for $\mathrm{L}^2(\mathbb{R}^D)$. The Galerkin approximation (see section 2.4 of
\cite{FGL_semiclassical_dynamics}) can then be written as:

\begin{quotation}
  \noindent
  At every time $t$ determine $\partial_t u \in \mathcal{M}$ such that
  \begin{equation*}
    \forall \vec{k} \in \mathfrak{K} \quad \langle \phi_{\vec{k}}, (i\varepsilon^2\partial_t -W) u \rangle = 0
  \end{equation*}
  with $u \in \mathcal{M}$.
\end{quotation}

This integral can now be split like:

\begin{align*}
  0 & = \langle \phi_{\vec{k}}, (i\varepsilon^2\partial_t -W) u \rangle \\
    & = \langle \phi_{\vec{k}}, i\varepsilon^2\partial_t u \rangle - \langle \phi_{\vec{k}}, W u \rangle
\end{align*}

and the solution $u$ can in turn be replaced by its basis expansion:

\begin{equation*}
  \langle \phi_{\vec{k}}, i\varepsilon^2 \partial_t
                          \sum_{\vec{l}\in\mathfrak{K}} c_{\vec{l}} \phi_{\vec{l}}
  \rangle -
  \langle \phi_{\vec{k}}, W
                          \sum_{\vec{l}\in\mathfrak{K}} c_{\vec{l}} \phi_{\vec{l}}
  \rangle = 0 \,.
\end{equation*}

Next we carry out the differentiation $\partial_t c_{\vec{l}} \phi_{\vec{l}}$.
We know that $c_{\vec{l}}(t)$ depends only on time and $\phi_{\vec{l}}[\Pi](\vec{x})$
depends only on space (ignoring the time dependence of $\Pi$) and whence we get:

\begin{equation*}
  \partial_t c_{\vec{l}} \phi_{\vec{l}} = \dot{c_{\vec{l}}} \phi_{\vec{l}} \,.
\end{equation*}

Plugging this into the integrals above we obtain:

\begin{equation*}
  \langle \phi_{\vec{k}}, i\varepsilon^2
                          \sum_{\vec{l}\in\mathfrak{K}} \dot{c_{\vec{l}}} \phi_{\vec{l}}
  \rangle -
  \langle \phi_{\vec{k}}, W
                          \sum_{\vec{l}\in\mathfrak{K}} c_{\vec{l}} \phi_{\vec{l}}
  \rangle = 0
\end{equation*}

and pulling out the summations and constants yields:

\begin{equation*}
  i\varepsilon^2 \sum_{\vec{l}\in\mathfrak{K}} \dot{c_{\vec{l}}}
  \langle \phi_{\vec{k}}, \phi_{\vec{l}} \rangle
  -
  \sum_{\vec{l}\in\mathfrak{K}} c_{\vec{l}}
  \langle \phi_{\vec{k}}, W \phi_{\vec{l}} \rangle
  = 0 \,.
\end{equation*}

The first integral vanishes by orthonormality of the basis functions and what
remains is:

\begin{equation*}
  i\varepsilon^2 \dot{c_{\vec{k}}}
  =
  \sum_{\vec{l}\in\mathfrak{K}} c_{\vec{l}}
  \langle \phi_{\vec{k}}, W \phi_{\vec{l}} \rangle
  \quad \forall \vec{k} \in \mathfrak{K} \,.
\end{equation*}

We can stack all the $|\mathfrak{K}|$ equations and get the following system
of coupled ordinary differential equations:

\begin{equation*}
  i\varepsilon^2
  \begin{pmatrix}
    \vdots \\
    \dot{c_{\vec{k}}} \\
    \vdots
  \end{pmatrix}
  =
  \begin{pmatrix}
    {}     & \vdots                                  & {} \\
    \cdots & \dotp{\phi_{\vec{k}}}{W \phi_{\vec{l}}} & \cdots \\
    {}     & \vdots                                  & {}
  \end{pmatrix}
  \begin{pmatrix}
    \vdots \\
    c_{\vec{l}} \\
    \vdots
  \end{pmatrix}
\end{equation*}

or in more compact matrix notation:

\begin{equation}
  \dot{\vec{c}} = -\frac{i}{\varepsilon^2} \mat{F} \vec{c} \,.
\end{equation}

The solution of this system is trivially given by:

\begin{equation} \label{eq:update_scheme_coeff}
  \vec{c}(t) = \exp\left(-\frac{i}{\varepsilon^2} t \mat{F}\right) \vec{c}(0) \,.
\end{equation}

This was the last missing bit for the time propagation of scalar wavepackets
$\Ket{\Phi}$ inside arbitrarily shaped potentials $V(\vec{x})$. In algorithm
\ref{al:tp_wave_packets_scalar} we assembled all the pieces and provide a pseudo
code of the implementation.

\begin{algorithm}
\caption{Time propagation of scalar wavepackets $\Ket{\Phi}$}
\label{al:tp_wave_packets_scalar}
\begin{algorithmic}
  \REQUIRE A semi-classical wavepacket $\Ket{\Phi\ofs{\vec{x},t}}$
  \REQUIRE The Hagedorn parameter set $\Pi$ and the coefficients $\{c_{\vec{k}}\}_{\vec{k}\in\mathfrak{K}}$
  \REQUIRE The basis shape $\mathfrak{K}$ and the mapping $\mu_{\mathfrak{K}}$
  \REQUIRE The time step $\tau$
  \STATE // Propagate with the kinetic operator
  \STATE $\vec{q} \assign \vec{q} + \frac{\tau}{2} \mat{M}\inv \vec{p}$
  \STATE $\mat{Q} \assign \mat{Q} + \frac{\tau}{2} \mat{M}\inv \mat{P}$
  \STATE $S \assign S + \frac{\tau}{4} \vec{p}\T \mat{M}\inv \vec{p}$
  \STATE // Propagate with the local quadratic potential
  \STATE $\vec{p} \assign \vec{p} - \tau \, \nabla V\ofs{\vec{q}}$
  \STATE $\mat{P} \assign \mat{P} - \tau \, \nabla^2 V\ofs{\vec{q}} \mat{Q}$
  \STATE $S \assign S - \tau \, V\ofs{\vec{q}}$
  \STATE // Propagate with the non-quadratic remainder $W$
  \STATE // Assemble the matrix $\mat{F}$
  \STATE $\mat{F}_{\mu(\vec{k}),\mu(\vec{l})} \assign
          \Braket{\phi_{\vec{k}} | W\ofs{\vec{x}} | \phi_{\vec{l}}} \quad \forall \vec{k}, \vec{l} \in \mathfrak{K}$
  \STATE // And propagate the coefficients
  \STATE $\vec{c} \assign \exp\ofs{-\tau \frac{i}{\varepsilon^2} \mat{F}} \vec{c}$
  \STATE // Propagate with the kinetic operator again
  \STATE $\vec{q} \assign \vec{q} + \frac{\tau}{2} \mat{M}\inv \vec{p}$
  \STATE $\mat{Q} \assign \mat{Q} + \frac{\tau}{2} \mat{M}\inv \mat{P}$
  \STATE $S \assign S + \frac{\tau}{4} \vec{p}\T \mat{M}\inv \vec{p}$
  \RETURN $\Ket{\Phi\ofs{\vec{x},t+\tau}}$
\end{algorithmic}
\end{algorithm}


\section{Vector valued wavepackets}


In the last section we reviewed the time propagation algorithm for scalar semi-classical
wavepackets $\Ket{\Phi}$ or equivalently for potentials with a single energy level only.
For simulations in the non-adiabatic case we need an extended version that handles
vectorial wavepackets $\Ket{\Psi}$ as defined in \eqref{eq:vectorial_wavepacket_hom}
and \eqref{eq:vectorial_wavepacket_inhom}. We examine the case of homogeneous
wavepackets first. This has mainly two reasons, namely homogeneous wavepackets
play a more important role in practical simulations and we can show the important
concepts easier while the inhomogeneous case can be treated by simple generalisation
later.


\subsection{Homogeneous wavepacket propagation}


As we defined in \eqref{eq:vectorial_wavepacket_hom} a homogeneous wavepacket
$\Psi$ consists of $N$ components $\Phi_i$ all sharing the same parameter set
$\Pi$. This set of parameters is propagated by the same rules as in the scalar
case. The propagation of $\Pi$ has to take place in the eigenbasis of $\mat{V}$
where the energy levels $\lambda_i$ decouple. We need to choose one of these
levels $\lambda_i(\vec{x})$ which then governs the propagation of $\Pi$ and
plays the role of $V$ in the scalar case. We denote this particular level
by $\lambda_{\chi}$ and call $\chi \in [0, \ldots, N-1]$ the \emph{characteristic
component index}. Then we apply the usual splitting into quadratic part
and non-quadratic remainder. Formally we write:

\begin{align*}
  u_\chi(\vec{x}) & = \lambda_\chi(\vec{x}) + \nabla \lambda_\chi(\vec{q}) (\vec{x}-\vec{q})
                      + \frac{1}{2} (\vec{x}-\vec{q})\T \nabla^2 \lambda_\chi(\vec{q}) (\vec{x}-\vec{q}) \\
  w_\chi(\vec{x}) & = \lambda_\chi(\vec{x}) - u_\chi(\vec{x}) \,.
\end{align*}

This is sufficient for time propagation of $\Pi$ but not for the Galerkin step
where we propagate the coefficients $\{c_{\vec{k}}\}_{\vec{k} \in \mathfrak{K}}$.
There we need a splitting of the full matrix $\mat{V}(\vec{x}) = \mat{U}(\vec{x}) + \mat{W}(\vec{x})$
into quadratic part $\mat{U}(\vec{x})$ and remainder $\mat{W}(\vec{x})$ such that:

\begin{equation*}
  \mat{V} =
  \begin{pmatrix}
    u_\chi & {}     & {} \\
    {}     & \ddots & {} \\
    {}     &        & u_\chi
  \end{pmatrix}
  +
    \begin{pmatrix}
    v_{0,0} - u_\chi & \cdots & v_{0,N-1} \\
    \vdots           & \ddots & \vdots \\
    v_{N-1,0}        & \cdots & v_{N-1,N-1} - u_\chi
  \end{pmatrix} \,.
\end{equation*}

We will need $\mat{W}$ for building the matrix $\mat{F}$ later. Notice that up
to now we did not mix the different components $\Phi_i$. This only happens
during propagation of the coefficients. For this we stack the coefficients
$\{c_{\vec{k}}^i\}_{\vec{k} \in \mathfrak{K}_i}$ of all components $\Phi_i$
into a long column vector $\vec{c}$:

\begin{equation*}
  \vec{c} \assign
  \begin{pmatrix}
    \cdots & c_{\vec{k}}^0 & \cdots & | & \cdots & | & \cdots & c_{\vec{k}}^{N-1} & \cdots
  \end{pmatrix}
  \T
\end{equation*}

of length $\sum_{i=0}^{N-1} |\mathfrak{K}_i|$ or $N |\mathfrak{K}|$ if all
components have a basis shape of same size. Then we build the $\mat{F}$ matrix
for use in equation \eqref{eq:update_scheme_coeff}. Obviously it must have the
following block structure:

\begin{equation*}
  \mat{F} \assign
  \begin{pmatrix}
    \mat{F_{0,0}}   & \cdots        & \mat{F_{0, N-1}} \\
    \vdots          & \mat{F_{i,j}} & \vdots \\
    \mat{F_{N-1,0}} & \cdots        & \mat{F_{N-1, N-1}} \\
  \end{pmatrix}
\end{equation*}

and each block is of the form:

\begin{equation*}
  \mat{F_{i,j}} \assign
    \begin{pmatrix}
    {}     & \vdots                                               & {} \\
    \cdots & \Braket{ \phi_{\vec{k}} | \mat{W}_{i,j} | \phi_{\vec{l}} } & \cdots \\
    {}     & \vdots                                               & {}
  \end{pmatrix}
\end{equation*}

for $\vec{k} \in \mathfrak{K}_i$ and $\vec{l} \in \mathfrak{K}_j$. These blocks
are not necessarily square but the matrix $\mat{F}$ always is. (Compare this
to the matrices used for the basis transformation of wavepackets.)

Algorithm \ref{al:build_block_matrix_homog} gives pseudo code for the computation
of $\mat{F}$ given a quadrature rule, the remainder $\mat{W}$ and the homogeneous
wavepacket $\Psi[\Pi]$. The time propagation then is summarised in algorithm
\ref{al:tp_wave_packets_homog}.


\begin{algorithm}
\caption{Build the homogeneous block matrix $\mat{F} \assign \left(\mat{F_{r,c}}\right)_{r,c}$}
\label{al:build_block_matrix_homog}
\begin{algorithmic}
  \REQUIRE A homogeneous wavepacket $\Psi$ with parameter set $\Pi$
  \REQUIRE The basis shape $\mathfrak{K}_i$ of each component $\Phi_i$ of $\Psi$, $i = 0,\ldots, N-1$
  \REQUIRE A $N \times N$ matrix $\mat{W}(\vec{x})$ of scalar functions $w_{r,c}(\vec{x})$
  \REQUIRE A quadrature rule $(\vec{\gamma}_j, \omega_j)$ with $R$ node-weight pairs
  \STATE // Initialise $\mat{F}$ as the zero-matrix
  \STATE $\eta \assign \sum_{i=0}^{N-1} |\mathfrak{K}_i|$
  \STATE $\mat{F} \in \mathbb{C}^{\eta \times \eta}, \quad \mathbf{F} \assign \mathbf{0}$
  \STATE // Transform the quadrature nodes according to \eqref{eq:transformed_qr_nodes}
  \STATE $\vec{\gamma}_j^\prime = \vec{q} + \varepsilon \, \mat{Q} \, \vec{\gamma}_j \quad j = 0, \ldots R-1$
  \STATE // Evaluate the basis functions of each component $\Phi_i$ with algorithm \ref{al:eval_phi_basis}
  \FOR{$i = 0$ \TO $i = N-1$}
    \STATE $\mat{B_i} \assign \mathbf{evaluate\_basis\_at}[\Phi_i]\left(\left(\vec{\gamma}_0^\prime, \ldots, \vec{\gamma}_{R-1}^\prime\right)\right)$
  \ENDFOR
  \STATE // Iterate over all row and column blocks of this matrix
  \FOR{$r = 0$ \TO $r = N-1$}
    \FOR{$c = 0$ \TO $c = N-1$}
      \STATE // Evaluate the function $w_{r,c}$ for all quadrature nodes $\vec{\gamma}_j$
      \STATE $\left(v_0, \ldots, v_{R-1}\right) \assign w_{r,c}\ofs{\left(\vec{\gamma}_0^\prime, \ldots, \vec{\gamma}_{R-1}^\prime\right)}$
      \STATE // Set up a zero matrix
      \STATE $\mat{F_{r,c}} \in \mathbb{C}^{|\mathfrak{K}_r| \times |\mathfrak{K}_c|}, \quad \mat{F_{r,c}} \assign \mat{0}$
      \STATE // Iterate over all $R$ quadrature pairs $\left(\vec{\gamma}^\prime_j, \omega_j\right)$
      \FOR{$j = 0$ \TO $j = R-1$}
        \STATE $\mat{F_{r,c}} \assign \mat{F_{r,c}} + \varepsilon^D \, v_{j} \, \omega_j \, \conj{\mat{B_r}[:,j]} \mat{B_c}[:,j]\T$
      \ENDFOR
      \STATE // Insert the block $\mat{F_{r,c}}$ into the block matrix $\mat{F}$
      \STATE $\mat{F}_{r,c} \assign \mat{F_{r,c}}$
    \ENDFOR
  \ENDFOR
  \RETURN $\mat{F}$
\end{algorithmic}
\end{algorithm}


\begin{algorithm}
\caption{Time propagation of a homogeneous wavepacket $\Ket{\Psi}$}
\label{al:tp_wave_packets_homog}
\begin{algorithmic}
  \REQUIRE A semi-classical wavepacket $\Ket{\Psi\ofs{\vec{x},t}}$
  \REQUIRE The corresponding Hagedorn parameter set $\Pi$
  \REQUIRE For all components $n \in [0, \ldots, N-1]$: the coefficients $\{c^n_{\vec{k}}\}_{\vec{k}\in\mathfrak{K}_n}$,
  \REQUIRE the basis shapes $\mathfrak{K}_n$ and the mappings $\mu_{\mathfrak{K}_n}$
  \REQUIRE The leading component index $\chi$
  \REQUIRE The time step $\tau$
  \STATE // Propagate with the kinetic operator
  \STATE $\vec{q} \assign \vec{q} + \frac{\tau}{2} \mat{M}\inv \vec{p}$
  \STATE $\mat{Q} \assign \mat{Q} + \frac{\tau}{2} \mat{M}\inv \mat{P}$
  \STATE $S \assign S + \frac{\tau}{4} \vec{p}\T \mat{M}\inv \vec{p}$
  \STATE // Propagate with the local quadratic potential
  \STATE $\vec{p} \assign \vec{p} - \tau \, \nabla \lambda_\chi\ofs{\vec{q}}$
  \STATE $\mat{P} \assign \mat{P} - \tau \, \nabla^2 \lambda_\chi\ofs{\vec{q}} \mat{Q}$
  \STATE $S \assign S - \tau \, \lambda_\chi\ofs{\vec{q}}$
  \STATE // Propagate with the non-quadratic remainder $\mat{W}$
  \STATE // Stack the coefficient vectors $\vec{c}^n$ of all components
  \STATE $\vec{C} \assign \left(\vec{c}^0 | \hdots | \vec{c}^{N-1}\right)\T$
  \STATE // Assemble the block matrix $\mat{F}$ using algorithm \ref{al:build_block_matrix_homog}
  \STATE $\Pi^\prime \assign \{\vec{q}, \vec{p}, \mat{Q}, \mat{P}, S\}$
  \STATE $\mat{F} \assign \mathbf{build\_homogeneous\_block\_matrix}(\mat{W}, \Psi[\Pi^\prime])$
  \STATE // Propagate the coefficients
  \STATE $\vec{C} \assign \exp\ofs{-\tau \frac{i}{\varepsilon^2} \mat{F}} \vec{C}$
  \STATE // Split the coefficients
  \STATE $\left(\vec{c}^0 | \ldots | \vec{c}^{N-1}\right) \assign \vec{C}$
  \STATE // Propagate with the kinetic operator again
  \STATE $\vec{q} \assign \vec{q} + \frac{\tau}{2} \mat{M}\inv \vec{p}$
  \STATE $\mat{Q} \assign \mat{Q} + \frac{\tau}{2} \mat{M}\inv \mat{P}$
  \STATE $S \assign S + \frac{\tau}{4} \vec{p}\T \mat{M}\inv \vec{p}$
  \RETURN $\Ket{\Psi\ofs{\vec{x}, t+\tau}}$
\end{algorithmic}
\end{algorithm}


\subsection{Inhomogeneous wavepacket propagation}


The case of inhomogeneous wavepacket propagation is very similar to the homogeneous
one presented in the last section. Here we only describe the differences that
are necessary. An inhomogeneous wavepacket $\Psi$ consists of $N$ components
$\Phi_i$ each having its own parameter set $\Pi_i$. It is self-evident that now
each energy level $\lambda_i(\vec{x})$ is used to propagate the corresponding
parameter set $\Pi_i$ of the component $\Phi_i$ that resides on this level.
Whence we have to apply the splitting to each eigenvalue:

\begin{align*}
  u_i(\vec{x}) & = \lambda_i(\vec{x}) + \nabla \lambda_i(\vec{q}^i) (\vec{x}-\vec{q}^i)
                      + \frac{1}{2} (\vec{x}-\vec{q}^i)\T \nabla^2 \lambda_i(\vec{q}^i) (\vec{x}-\vec{q}^i) \\
  w_i(\vec{x}) & = \lambda_i(\vec{x}) - u_i(\vec{x}) \,.
\end{align*}

for all $i \in [0, \ldots, N-1]$ and $\vec{q}^i \in \Pi_i$. This covers all we
need to propagate the parameter sets $\{ \Pi_i \}_{i=0}^{N-1}$. The splitting of
the whole potential matrix $\mat{V}$ into $\mat{U}$ and $\mat{W}$ is straight forward:

\begin{equation*}
  \mat{V} =
  \begin{pmatrix}
    u_0 & {}     & {} \\
    {}  & \ddots & {} \\
    {}  &        & u_{N-1}
  \end{pmatrix}
  +
    \begin{pmatrix}
    v_{0,0} - u_0 & \cdots & v_{0,N-1} \\
    \vdots        & \ddots & \vdots \\
    v_{N-1,0}     & \cdots & v_{N-1,N-1} - u_{N-1}
  \end{pmatrix} \,.
\end{equation*}

Again we use the non-quadratic remainder $\mat{W}(\vec{x})$ to compute the block
matrix $\mat{F}$. The main difference here lies inside the off-diagonal blocks. While
building $\mat{F_{i,j}}$ with $i \neq j$ the functions $\phi_{\vec{k}}$
and $\phi_{\vec{l}}$ appearing there belong to different families with in general
different parameter sets $\Pi_i$ and $\Pi_j$. Therefore the correct formula for
these blocks reads:

\begin{equation*}
  \mat{F_{i,j}} \assign
    \begin{pmatrix}
    {}     & \vdots                                                                  & {} \\
    \cdots & \Braket{ \phi_{\vec{k}}[\Pi_i] | \mat{W}_{i,j} | \phi_{\vec{l}}[\Pi_j] } & \cdots \\
    {}     & \vdots                                                                  & {}
  \end{pmatrix}
\end{equation*}

where $\vec{k} \in \mathfrak{K}_i$ and $\vec{l} \in \mathfrak{K}_j$. The block is
of size $|\mathfrak{K}_i| \times |\mathfrak{K}_j|$. To compute the entries we are
required to use an inhomogeneous quadrature rule.

Everything else works exactly the same as in the homogeneous case. Pseudo code
for building the block matrix \ref{al:build_block_matrix_inhomog} and for the
time propagation \ref{al:tp_wave_packets_inhomog} is presented below for explicit
reference.


\begin{algorithm}
\caption{Build the inhomogeneous block matrix $\mat{F} \assign \left(\mat{F_{r,c}}\right)_{r,c}$}
\label{al:build_block_matrix_inhomog}
\begin{algorithmic}
  \REQUIRE An inhomogeneous wavepacket $\Psi$ with $N$ parameter sets $\Pi_i$
  \REQUIRE The basis shape $\mathfrak{K}_i$ of each component $\Phi_i$ of $\Psi$, $i = 0,\ldots, N-1$
  \REQUIRE A $N \times N$ matrix $\mat{W}(\vec{x})$ of scalar functions $w_{r,c}(\vec{x})$
  \REQUIRE A quadrature rule $(\vec{\gamma}_j, \omega_j)$ with $R$ node-weight pairs
  \STATE // Initialise $\mat{F}$ as the zero-matrix
  \STATE $\eta \assign \sum_{i=0}^{N-1} |\mathfrak{K}_i|$
  \STATE $\mat{F} \in \mathbb{C}^{\eta \times \eta}, \quad \mathbf{F} \assign \mathbf{0}$
  \STATE // Iterate over all row and column blocks of this matrix
  \FOR{$r = 0$ \TO $r = N-1$}
    \FOR{$c = 0$ \TO $c = N-1$}
      \STATE // Apply the mixing formula from procedure \ref{al:mixing_hagedorn_parameters} to the parameters
      \STATE $\vec{q_0}, \mat{Q_S} \assign \mathbf{mix\_parameters}(\Pi_r, \Pi_c)$
      \STATE // Transform the quadrature nodes according to \eqref{eq:transform_qr_nodes}
      \STATE $\vec{\gamma}_j^\prime = \vec{q_0} + \varepsilon \, \mat{Q_S} \, \vec{\gamma}_j \quad j = 0, \ldots R-1$
      \STATE // Evaluate the function $w_{r,c}$ for all quadrature nodes $\vec{\gamma}^\prime$
      \STATE $\left(v_0, \ldots, v_{R-1}\right) \assign w_{r,c}\ofs{\left(\vec{\gamma}^\prime_0, \ldots, \vec{\gamma}^\prime_{R-1}\right)}$
      \STATE // Evaluate the basis functions for all quadrature nodes $\vec{\gamma}^\prime$
      \STATE // Apply algorithm \ref{al:eval_phi_basis} to evaluate the basis of $\Phi_r$ and $\Phi_c$
      \STATE $\mat{B_r} \assign \mathbf{evaluate\_basis\_at}[\Phi_r]\left(\left(\vec{\gamma}_0^\prime, \ldots, \vec{\gamma}_{R-1}^\prime\right)\right)$
      \STATE $\mat{B_c} \assign \mathbf{evaluate\_basis\_at}[\Phi_c]\left(\left(\vec{\gamma}_0^\prime, \ldots, \vec{\gamma}_{R-1}^\prime\right)\right)$
      \STATE // Do not forget the non-vanishing phase
      \STATE $\pi_{r,c} \assign \exp\ofs{\frac{i}{\varepsilon^2}\left(S_c - \conj{S_r}\right)}$
      \STATE // Set up a zero matrix
      \STATE $\mat{F_{r,c}} \in \mathbb{C}^{|\mathfrak{K}_r| \times |\mathfrak{K}_c|}, \quad \mat{F_{r,c}} \assign \mat{0}$
      \STATE // Iterate over all $R$ quadrature pairs $\left(\vec{\gamma}^\prime_j, \omega_j\right)$
      \FOR{$j = 0$ \TO $j = R-1$}
        \STATE $\mat{F_{r,c}} \assign \mat{F_{r,c}} + \varepsilon^D \, v_j \, \omega_j \, \det\left(\mat{Q_S}\right) \, \conj{\mat{B_r}[:,j]} \mat{B_c}[:,j]\T$
      \ENDFOR
      \STATE // Insert the block $\mat{F_{r,c}}$ into the block matrix $\mat{F}$
      \STATE $\mat{F}_{r,c} \assign \pi_{r,c} \, \mat{F_{r,c}}$
    \ENDFOR
  \ENDFOR
  \RETURN $\mat{F}$
\end{algorithmic}
\end{algorithm}


\begin{algorithm}
\caption{Time propagation of an inhomogeneous wavepacket $\Ket{\Psi}$}
\label{al:tp_wave_packets_inhomog}
\begin{algorithmic}
  \REQUIRE A semi-classical wavepacket $\Ket{\Psi\ofs{\vec{x},t}}$
  \REQUIRE For all components $n \in [0, \ldots, N-1]$:
  \REQUIRE the Hagedorn parameter sets $\Pi_n$, the coefficients $\{c^n_{\vec{k}}\}_{\vec{k}\in\mathfrak{K}_n}$
  \REQUIRE the basis shapes $\mathfrak{K}_n$ and the mappings $\mu_{\mathfrak{K}_n}$
  \REQUIRE The time step $\tau$
  \STATE // Propagate with the kinetic operator
  \FOR{$n = 0$ \TO $n = N-1$}
    \STATE $\vec{q}_n \assign \vec{q}_n + \frac{\tau}{2} \mat{M}\inv \vec{p}_n$
    \STATE $\mat{Q}_n \assign \mat{Q}_n + \frac{\tau}{2} \mat{M}\inv \mat{P}_n$
    \STATE $S_n \assign S_n + \frac{\tau}{4} \vec{p}_n\T \mat{M}\inv \vec{p}_n$
  \ENDFOR
  \STATE // Propagate with the local quadratic potential
  \FOR{$n = 0$ \TO $n = N-1$}
    \STATE $\vec{p}_n \assign \vec{p}_n - \tau \, \nabla \lambda_n\ofs{\vec{q}_n}$
    \STATE $\mat{P}_n \assign \mat{P}_n - \tau \, \nabla^2 \lambda_n\ofs{\vec{q}_n} \mat{Q}_n$
    \STATE $S_n \assign S_n - \tau \, \lambda_n\ofs{\vec{q}_n}$
  \ENDFOR
  \STATE // Propagate with the non-quadratic remainder
  \STATE // Stack the coefficient vectors $\vec{c}^n$ of all components
  \STATE $\vec{C} \assign \left(\vec{c}^0 | \hdots | \vec{c}^{N-1}\right)\T$

  \STATE // Assemble the matrix $\mat{F}$ using algorithm \ref{al:build_block_matrix_inhomog}
  \STATE $\Pi_i^\prime \assign \{\vec{q}_n, \vec{p}_n, \mat{Q}_n, \mat{P}_n, S_n\} \quad \forall n = 0, \ldots ,N-1$
  \STATE $\mat{F} \assign \mathbf{build\_inhomogeneous\_block\_matrix}(\mat{W}, \Psi[\Pi^\prime_0, \ldots ,\Pi^\prime_{N-1}])$

  \STATE // Propagate the coefficients
  \STATE $\vec{C} \assign \exp\ofs{-\tau \frac{i}{\varepsilon^2} \mat{F}} \vec{C}$
  \STATE // Split the coefficients
  \STATE $\left(\vec{c}^0 | \ldots | \vec{c}^{N-1}\right) \assign \vec{C}$

  \STATE // Propagate with the kinetic operator again
  \FOR{$n = 0$ \TO $n = N-1$}
    \STATE $\vec{q}_n \assign \vec{q}_n + \frac{\tau}{2} \mat{M}\inv \vec{p}_n$
    \STATE $\mat{Q}_n \assign \mat{Q}_n + \frac{\tau}{2} \mat{M}\inv \mat{P}_n$
    \STATE $S_n \assign S_n + \frac{\tau}{4} \vec{p}_n\T \mat{M}\inv \vec{p}_n$
  \ENDFOR
  \RETURN $\Ket{\Psi\ofs{\vec{x}, t+\tau}}$
\end{algorithmic}
\end{algorithm}
