\section{Inner Products}

In order to calculate wavepacket norms, various energies and other observables,
it is necessary to evaluate the braket:
\begin{equation}
  \label{eq:braket}
  \mat{M} = \langle \Psi | F | \Psi' \rangle,
\end{equation}
where $\Psi$ and $\Psi'$ are wavepackets and $F$ specifies the operator.

For the mathematical derivation of the braket evaluation algorithms, again refer
to \cite{B_master_thesis}.


\subsection{Scalar Inner Products}

In the case of scalar wavepackets $\Phi[\Pi]$ and $\Phi'[\Pi']$ the braket is written as:
\begin{equation}
  \mat{M} = \langle \Phi[\Pi] | f | \Phi'[\Pi'] \rangle.
\end{equation}
The operator $f$ is given as a function parameter of the form:
\begin{equation}
  f(\mat{x}, \vect{q}) : \mathbb{C}^{D \times R} \times \mathbb{R}^D \rightarrow
  \mathbb{C}^R,
\end{equation}
mapping nodal points $\vect{x} \in \mathbb{R}^D$ and position
$\vect{q} \in \mathbb{R}^D$ to the operator's values.

Here, $R$ is the order of the quadrature.
Instead of calling the operator function for the different nodes $\vect{x}$
separately, all points are given to the function at once as a matrix $\mat{x}
\in \mathbb{C}^{D \times R}$ to make efficient vectorized calculations in its
definition possible.

The inner product calculation implementation provides two different methods,
\texttt{build\_matrix} and \texttt{quadrature}.
The first returns a $|\mathfrak{K}| \times |\mathfrak{K'}|$-sized matrix, one
entry for each pair of basis functions of $\Phi[\Pi]$ and $\Phi'[\Pi']$, where
$|\mathfrak{K}|$ is the basis shape size.
\texttt{quadrature} reduces the matrix to a scalar by multiplying with the
coefficient vectors $\vect{c}$ and $\vect{c}'$ of $\Phi[\Pi]$ and $\Phi'[\Pi']$, returning
$\vect{c}^H \mat{M} \vect{c}'$.

The ``inhomogeneous'' evaluation code for different wavepackets $\Phi[\Pi]$
and $\Phi'[\Pi']$ can be found in
\path{waveblocks/inhomogeneous_inner_product.hpp}.
If the two wavepackets are equivalent ($\Phi[\Pi] = \Phi'[\Pi']$), a more efficient
``homogeneous'' algorithm can be used instead which is implemented in
\path{waveblocks/homogeneous_inner_product.hpp}


\subsection{Multi-Component Inner Products}

In the more general case the wavepackets can have multiple components:
\begin{align}
  \Psi &= \begin{pmatrix} \Phi_1[\Pi_1] \\ \vdots \\ \Phi_N[\Pi_N] \end{pmatrix} \\
  \mat{M} &= \langle \Psi | F | \Psi' \rangle = \left\langle
    \begin{pmatrix} \Phi_1[\Pi_1] \\ \vdots \\ \Phi_N[\Pi_N] \end{pmatrix} \left|
    \begin{pmatrix} \ddots & \vdots & \udots \\ \cdots & f_{i,j} & \cdots \\
      \udots & \vdots & \ddots \end{pmatrix} \right|
    \begin{pmatrix} \Phi'_1[\Pi_1'] \\ \vdots \\ \Phi'_{N'}[\Pi_{N'}'] \end{pmatrix}
    \right\rangle \,.
\end{align}
Instead of passing a matrix of functions for the operator $F$ like in the
formula above, the code expects one function parameter of the following type:
\begin{equation}
  f(\mat{x}, \vect{q}, i, j) : \mathbb{C}^{D \times R} \times \mathbb{R}^D
  \times \mathbb{N} \times \mathbb{N} \rightarrow \mathbb{C}^R \,.
\end{equation}
The calculation of these brakets is built on top of the inhomogeneous scalar
evaluation code, calling it for every pair of wavepackets $(\Phi_i,\Phi'_j)$ and
corresponding function $f_{i,j}$.

In the implementation, the resulting matrix $\mat{M}$ is built up of $N \times
N'$ blocks, each of size $|\mathfrak{K}_i| \times |\mathfrak{K}_j'|$:
\begin{equation}
  \mat{M} = \begin{pmatrix} \ddots & \vdots & \udots \\
    \cdots & \mat{M}_{i,j} & \cdots \\
    \udots & \vdots & \ddots
  \end{pmatrix} =
  \begin{pmatrix} \ddots & \vdots & \udots \\
    \cdots & \langle \Phi_i[\Pi_i]|f_{i,j}|\Phi'_j[\Pi_j'] \rangle & \cdots \\
    \udots & \vdots & \ddots
  \end{pmatrix} \,.
\end{equation}
If quadrature is requested, a vector of $N \cdot N'$ scalars is returned, with
components $\vect{c}_i^H \ \mat{M}_{i,j} \ \vect{c}'_j$.

The source code for this algorithm is found in
\path{waveblocks/vector_inner_product.hpp}.
