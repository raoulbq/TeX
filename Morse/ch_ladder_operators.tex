\begin{chapter}{Ladder operators for the Morse Potential}
In this chapter we derive the explicit form of the ladder operators for the Morse eigenfunctions,
following \cite{Dong_ladder_operators}.\\
As already mentioned in \ref{sec:morse_potential}, the Morse potential allows for a continuum of free states as well as a finite number of discrete bound states. In our case, only the bound 
solutions are of interest.


\section{Construction} % (fold)
\label{sec:Construction}
Dong starts by looking for differential operators of the form
\begin{equation}
    \label{eq:dong_operator_form}
    \hat{\mathcal{K}}_{\pm}=A_{\pm}(z)\diff{}{z}+B_{\pm}(z)
\end{equation}
with the property that
\begin{align}
    \hat{\mathcal{K}}_{\pm}\mu_n(z)&=k_{\pm}\mu_{n\pm 1}(z)\\
	\hat{\mathcal{K}}_{-}\mu_0&\equiv 0\\    
    \hat{\mathcal{K}}_{+}\mu_{\text{max}}&\equiv 0\;,
\end{align}
where $\mu_0$ and $\mu_{\text{max}}$ are referring to the lowest and highest Morse eigenfunction,
respectively.\\

Application of the differential operator $\mathrm{d}/\mathrm{d}z$ to the $n$-th Morse eigenfunction \eqref{eq:morse_function} leads to
\begin{equation}
    \label{eq:dong_apply_ddy}
    \diff{}{z}\mu_n(z)=-\frac{1}{2}\mu_n(z)+\frac{1}{z}s\mu_n(z)+N_n e^{-z/2}z^s\diff{}{z}\mathrm{L}_n^{2s}(z).
\end{equation}
where $\mathrm{L}_n^{2s}(z)$ denotes the n-th generalized Laguerre polynomial.\\

Using the recursion relation
\begin{equation}
	\label{eq:dong_ddy_rek}
	\diff{}{z}\mathrm{L}_{n}^\alpha(z)=-\frac{1}{\alpha+1}\left(z\mathrm{L}_{n-1}^{\alpha+2}(z)+n\mathrm{L}_{n}^\alpha(z)\right)
\end{equation}
taken from \cite{DongFactMethods} and substituting it into \eqref{eq:dong_apply_ddy} yields
\begin{equation}
    \left(\diff{}{z}(2s+1)-\left(\frac{1}{z}s-\frac{1}{2}\right)(2s+1)+n\right)\mu_n(z)=-\frac{N_n}{N_{n-1}}\mu_{n-1}(z) \;,
\end{equation}
such that the lowering operator\footnote{
    In \cite{Dong_ladder_operators} these operators are denoted as $\hat{K}_{\pm}$, while we choose to follow the notation in \cite{FGL_semiclassical_dynamics} 
    and use $\mathcal{L}$ for the lowering and $\mathcal{R}$ for the raising operator, respectively.}
can be defined as
\begin{equation}
    \label{eq:dong_lop}
    \mathcal{L}:=-\left(\diff{}{z}(2s+1)-\frac{1}{z}s(2s+1)+\frac{\nu}{2} \right)\sqrt{\frac{s+1}{s}}
\end{equation}
whose action on a Morse eigenfunction $\mu_n$ is given by
\begin{equation}
    \label{eq:dong_lop_action}
    \mathcal{L}\mu_n(z)=l\mu_{n-1}(z),\quad l=\sqrt{n(\nu-n)}\; .
\end{equation}

For the raising operator using \eqref{eq:dong_apply_ddy} again as well as the relations
\begin{align}
    &    z\diff{}{z}\mathrm{L}_n^\alpha(z)=n\mathrm{L}_n^{\alpha}(z)-(n+\alpha)\mathrm{L}_{n-1}^{\alpha}(z)\\
    &	(n+1)\mathrm{L}_{n+1}^{\alpha}(z)-(2n+\alpha+1-z)\mathrm{L}_n^{\alpha}(z)+(n+\alpha)\mathrm{L}_{n-1}^{\alpha}(z)=0\\
    &	\mathrm{L}_n^{\alpha-1}(z)=\mathrm{L}_n^{\alpha}(z)-\mathrm{L}_{n-1}^{\alpha}(z)
\end{align}
the following expression is found
\begin{equation}
    \label{eq:dong_rop}
    \mathcal{R}:=\left(\diff{}{z}(2s-1)+\frac{1}{z}s(2s-1)-\frac{\nu}{2} \right)\sqrt{\frac{s-1}{s}}
\end{equation}
with the corresponding action
\begin{equation}
    \label{eq:dong_rop_action}
    \mathcal{R}\mu_n(z)=r\mu_{n+1}(z),\quad r=\sqrt{(n+1)(\nu-n-1)}\; .
\end{equation}
% section Construction (end)

\section{Lie Algebra} % (fold)
\label{sec:LieAlgebra}
Next, calculating the commutator of the two ladder operators yields
\begin{equation}
    \label{eq:dong_commutator}
    [\mathcal{R},\mathcal{L}]\mu_n(z)=2k_0\mu_n(z)
\end{equation}
with eigenvalue
\begin{equation}
    \label{eq:dong_k0_ev}
    k_0=n-\frac{\nu-1}{2}\;,
\end{equation}
which can be rearranged to yield
\begin{equation}
\label{eq:dong_k0_ev2}
\frac{\nu}{2}=n-k_0-\frac{1}{2}\;.
\end{equation}

This can be plugged into the differential equation for the Morse potential \eqref{eq:morse_transformed_tdse} such that we obtain
\begin{equation}
\left( z\frac{\mathrm{d}^{2}}{\mathrm{d}z^{2}}+\diff{}{z}-\frac{s^2}{z}-\frac{z}{4}+n+\frac{1}{2} 
-k_0 \right)\mu_n=0\; .
\end{equation}

If we define the operator $\mathcal{K}_0$ as
\begin{equation}
    \mathcal{K}_0:=\left( z\frac{\mathrm{d}^{2}}{\mathrm{d}z^{2}}+\diff{}{z}-\frac{s^2}{z}-\frac{z}{4}+n+\frac{1}{2}  \right)\;
\end{equation}
we get the following eigenvalue equation
\begin{equation}
\label{eq:dong_k0eveq}
\mathcal{K}_0\mu_n=k_0\mu_n\; .
\end{equation}

Finally, comparing \eqref{eq:dong_k0eveq} and \eqref{eq:dong_commutator}, we can identify
$\mathcal{K}_0$ with the commutator $[\mathcal{R},\mathcal{L}]$ of $\mathcal{R}$ and $\mathcal{L}$. These three operators form a closed Lie algebra, which by their commutation relations
\begin{align}
    \begin{split}
	[\mathcal{R},\mathcal{L}]=2\mathcal{K}_0,\quad  [\mathcal{K}_0,\mathcal{L}]=-\mathcal{L},\quad
	[\mathcal{K}_0,\mathcal{R}]=\mathcal{R}      
    \end{split}
\end{align}
can also be identified as being isomorphic to the angular momentum algebra $\mathfrak{su}(2)$.\\

Finally, as we have seen in section \ref{sub:DiagonalizationHGHam}, an essential step towards the implementation of a time stepping algorithm is the 
factorization of the Hamiltonian using ladder operators because this will give us the connection to the wave packet parameter evolution.\\

In terms of this algebra of $SU(2)$ the Hamiltonian can be written as:
\begin{equation}
    \label{eq:DongHamiltonian}
    \mathcal{H}=-\frac{\hbar \omega}{\nu}\mathcal{K}_0^2\;,\qquad \omega=\frac{\hbar\beta^2\nu}{2\mu}
\end{equation}

This looks quite compact, but expanding the square and the commutator in terms of the ladder operators $\mathcal{R}$ and $\mathcal{L}$ 
we unfortunately get something more complicated than the simple anti-commutator expression \eqref{eq:HG_OpHamiltonian} in the Hagedorn case.


\section{Three-Term Recursion} % (fold)
\label{sec:ThreeTermRecursion}
Following the procedure in \cite{FGL_semiclassical_dynamics}, we will use \eqref{eq:dong_lop} and \eqref{eq:dong_rop} to obtain a three-term recursion
relation for the Morse eigenfunctions $\mu_n$. This relation is used for the fast and stable numerical evaluation of the Morse functions.\\

As a first step we can express the differential operator $\mathrm{d}/\mathrm{d}z$ in terms of our ladder operators $\mathcal{L}$ and $\mathcal{R}$:
\begin{equation}
    \label{eq:dong_ddyinLR}
    \diff{}{z}=\mathcal{R}\left(\frac{1}{2(2s-1)}\sqrt{\frac{s}{s-1}}\right)-\mathcal{L}\left(\frac{1}{2(2s+1)}\sqrt{\frac{s}{s+1}}\right)
    +\frac{\nu}{2(2s+1)(2s-1)}
\end{equation}
We then can eliminate this operator from \eqref{eq:dong_lop} and \eqref{eq:dong_rop} such that we get
\begin{align}
    \mathcal{L}=-\sqrt{\frac{s+1}{s-1}}\frac{2s+1}{2s-1}\mathcal{R}
    +\sqrt{\frac{s+1}{s}}\frac{2s}{2s-1}\left(\frac{4s^2-1}{z}-\nu\right)\label{eq:dong_lop2}\\
    \mathcal{R}=-\sqrt{\frac{s-1}{s+1}}\frac{2s-1}{2s+1}\mathcal{L}
    +\sqrt{\frac{s-1}{s}}\frac{2s}{2s+1}\left(\frac{4s^2-1}{z}-\nu\right)\label{eq:dong_rop2}
\end{align}

Next, reordering \eqref{eq:dong_rop_action} and substituting \eqref{eq:dong_rop2} yields
\begin{align}
    \mu_{n+1} &= \frac{1}{r}\mathcal{R}\mu_n\\
    &=\frac{1}{r}\left(-\sqrt{\frac{s-1}{s+1}}\frac{2s-1}{2s+1}\mathcal{L}
	+\sqrt{\frac{s-1}{s}}\frac{2s}{2s+1}\left(\frac{4s^2-1}{z}-\nu\right)\right)\mu_n
\end{align}
and finally, after substituting \eqref{eq:dong_lop_action} we obtain the desired three-term relation
\begin{equation}
    \mu_{n+1}=\frac{1}{r}\sqrt{\frac{s-1}{s}}\frac{2s}{2s+1}\left(\frac{4s^2-1}{z}-\nu\right)\mu_n
    -\frac{l}{r}\sqrt{\frac{s-1}{s+1}}\frac{2s-1}{2s+1}\mu_{n-1}
\end{equation}

The n-th state can then be explicitly written as
\begin{equation}
\mu_n(z)=\mathcal{N}_n\mathcal{R}^n\mu_0(z),\quad \mathcal{N}_n:=\sqrt{\frac{(\nu-n-1)!}{n!(\nu-1)!}}
\end{equation}

% section Three-Term Recursion (end)

\section{Back Transformation to original position space} % (fold)
\label{sec:Back_transf}
Our original physical position variable was exponentially rescaled in order to solve the differential equation and construct the ladder operators.
However, for the numerical treatment, especially the time stepping, this might cause problems.
Hence, we also provide a representation of the operators in the original $x$ domain.\\

Using the transformation $z=\nu\exp{(-\beta x)}$ we obtain the following transformed expression for the differential operator
\begin{equation}
    \label{eq:ddy}
    \diff{}{x}=-\beta z\diff{}{z}\;,
\end{equation}
such that the back transformation of the $\mathcal{L}$ and $\mathcal{R}$ operators is straightforward
\begin{align}
    \label{eq:dong_lrop_x}
    \begin{split}
    \mathcal{L}=\left(\frac{e^{\beta x}}{\beta\nu}\diff{}{x}(2s+1)+\frac{e^{\beta x}}{\nu}s(2s+1)-\frac{\nu}{2} \right)\sqrt{\frac{s+1}{s}}\\
    \mathcal{R}=\left(-\frac{e^{\beta x}}{\beta\nu}\diff{}{x}(2s-1)+\frac{e^{\beta x}}{\nu}s(2s-1)-\frac{\nu}{2} \right)\sqrt{\frac{s-1}{s}}
    \end{split}
\end{align}


% section Back Transformation to original position space (end)

\section{Matrix Elements} % (fold)
\label{sec:Matrix Elements}
Finally, to obtain any information about position and momentum, we need expressions for the matrix elements of the wave function with respect to various operators, the most important being position $\hat{x}$ and momentum $\hat{p}$.\\

In the Hagedorn case we have seen that the wave packet parameters $q$ and $p$ can be seen as mean position and momentum of the wave packet, while
$Q$ and $P$ describe its width in position and momentum space, respectively.\\

Hence, calculating the corresponding matrix elements within the ladder operator
formalism might give insights about the form of and relations among the wave packet parameters we are trying to find for the Morse wave packets.\\

To start with, the derivation of the three-term recursion in section \ref{sec:ThreeTermRecursion} already provides us with an expression for
the differential operator $\mathrm{d}/\mathrm{d}z$. In particular, in \eqref{eq:dong_ddyinLR} we expressed this operator solely in terms of
our newly constructed ladder operators such that the matrix element evaluates to

\begin{align}
    \begin{split}
	\braket{\mu_m\left|\diff{}{z}\right|\mu_n} &= \frac{1}{2(\nu-2n-2)}\sqrt{\frac{(n+1)(\nu-n-1)(\nu-2n-1)}{\nu-2n-3}}\delta_{m,n+1}\\
	&-\frac{1}{2(\nu-2n)}\sqrt{\frac{n(\nu-n)(\nu-2n-1)}{\nu-2n+1}}\delta_{m,n-1}\\
	&+\frac{\nu}{2(\nu-2n)(\nu-2n-2)}\delta_{m,n}\; .
    \end{split}
\end{align}

Repeating the procedure by which we obtained \eqref{eq:dong_ddyinLR} but eliminating $\mathrm{d}/\mathrm{d}z$ yields
\begin{equation}
    \label{eq:dong_1overyinLR}
    \frac{1}{z}=\mathcal{R}\left(\frac{1}{2s(2s-1)}\sqrt{\frac{s}{s-1}}\right)+\mathcal{L}\left(\frac{1}{2s(2s+1)}\sqrt{\frac{s}{s+1}}\right)
    +\frac{\nu}{(2s+1)(2s-1)}
\end{equation}
and the corresponding matrix element

\begin{align}
    \begin{split}
	\braket{\mu_m\left|\frac{1}{z}\right|\mu_n} &= \frac{1}{\nu-2n-2}\sqrt{\frac{(n+1)(\nu-n-1)}{(\nu-2n-1)(\nu-2n-3)}}\delta_{m,n+1}\\
	&+\frac{1}{\nu-2n}\sqrt{\frac{n(\nu-n)}{(\nu-2n-1)(\nu-2n+1)}}\delta_{m,n-1}\\
	&+\frac{\nu}{(\nu-2n)(\nu-2n-2)}\delta_{m,n}\;.
    \end{split}
\end{align}

For the position operator $\hat{x}$ we can use the following identity from \cite{DongFactMethods}
\begin{equation}
    \log z=\sum_{n=1}^\infty\frac{1}{n}\left(1-\frac{1}{z}\right)^n
\end{equation}
to express $\hat{x}$ in terms of the ladder operators:
\begin{align}
    \begin{split}
    \hat{x}&=-\frac{1}{\beta}\log\frac{z}{\nu}=\frac{1}{\beta}\left(\log\nu - \log z\right) 
    	=\frac{\log\nu}{\beta}
    		-\frac{1}{\beta}\sum_{n=1}^\infty\frac{1}{n}\left( 1-\frac{1}{z}\right)^n \\
	    &= -\frac{1}{\beta}\sum_{n=1}^\infty\frac{1}{n}\left( 1-
	    \mathcal{R}\left(\frac{1}{2s(2s-1)}\sqrt{\frac{s}{s-1}}\right)+\mathcal{L}\left(\frac{1}{2s(2s+1)}\sqrt{\frac{s}{s+1}}\right)
    +\frac{\nu}{(2s+1)(2s-1)}
	     \right)^n\\
	     +& \frac{\log\nu}{\beta}
    \end{split}
\end{align}

which is unfortunately not a closed form expression and as complicated as the expression we already derived in section \ref{sec:Moments} for the special case of a diagonal matrix element. Additional issues that need to be addressed are convergence of this series and where to truncate it in a
numerical computation.

% section Matrix Elements (end)


\end{chapter}
