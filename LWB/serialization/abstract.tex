\begin{abstract}
It is of general interest in physics and chemistry to find viable algorithms to solve the time-dependent Schr\"odinger equation. To solve this equation directly is in most cases not feasible so different approximations have to be made. Such an approximation is the semiclassical approach with \textit{Hagedorn} wavepackets. To test this approach for reliability a python implementation was created. With this implementation it is possible to easily test many starting configurations which results in a lot of data. To handle this huge amount of data a binary format is required because only this format can be efficiently compressed. Hence it is a good idea to use a well-known and commonly used binary data format such as \textit{HDF5}. With \textit{h5py} there exists also a reasonable python interface. Since not all problems are solvable in a short period of time, and also due to the fact that Python is an interpreted language, the code was ported to C++ to reduce execution time. Therefore also the IO-operations have to be ported to C++. The \textit{HDF5} library supports also a C++ interface. This however is not as easy comprehensible as the Python interface and thus this thesis explains its core functionality with respect to this application. Furthermore since the same data hierarchy is implemented as in Python the generated data is directly comparable. This data test was done in C++ with the well-known \textit{GoogleTest} framework, which can compare two data files generated from Python and C++ respectively.
\end{abstract}
