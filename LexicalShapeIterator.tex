\documentclass{article}
\usepackage{mathtools}
\usepackage{kpfonts}
\usepackage{listings}
\begin{document}

\section{Shape}

A \(D\)-dimensional shape \(\mathfrak{K}\) is a set of ordered D-dimensional integer-tuples.
A shape is suitable for our needs if it satisfies:
\[k \in \mathfrak{K} \Rightarrow \forall k^\star \preceq k \colon k^\star \in \mathfrak{K}\]
That means, if an arbitrary node is part of the shape, then all nodes in the backward cone are part of the shape too.

\section{Shape Enumeration}

A shape enumeration orders all nodes of a and assigns the \(i\)-th node the \emph{ordinal} \(i-1\).
Since many algorithm operating on wavepackets use a recursive formula, a shape enumeration is organized into \emph{slices}. The \(s\)-th slice of a shape \(\mathfrak{K}\) contains all nodes \(\boldsymbol{k} \in \mathfrak{K} \) that satisfy \(\sum_{d=1}^{D} k_D = s\).
\\ \\
Although no algorithm so far relies on a particular ordering of nodes inside a slice, it is probably wise to use lexical ordering by comparing element indices from left (first entry) to right (last entry).

\subsection{Abstract Shape Enumeration}

The file \emph{waveblocks/shape\_enumeration\_base.hpp} provides a interface that is meant to be overridden by different implementations. To represent a multi-index, this interface uses the greatest common denominator \emph{std::array\textless int,D\textgreater}. 

\subsection{Default Shape Enumeration}

The file \emph{waveblocks/shape\_enumeration\_default.hpp} provides a default implementation of a shape enumeration.

\begin{lstlisting}[language=C++, caption={}]
template<dim_t D, class MultiIndex, class S>
class DefaultShapeEnumeration;
\end{lstlisting}

\begin{description}
\item[D] number of dimensions
\item[MultiIndex] type to internally represent multi-indices
\item[S] shape description
\end{description}

\subsubsection{Internal multi-index representation}
The default shape enumeration stores all nodes in a vector. Since using \emph{std::array\textless int,D\textgreater} would use a lot of memory, the default enumeration exposes a template parameter to define a better type. \\ \\ A suitable type must \dots
\begin{itemize}
\item provide the same semantics as \emph{std::array\textless int,D\textgreater}
\item specialize \emph{std::less} to perform lexical comparison
\item specialize \emph{std::equal\_to}
\item specialize \emph{std::hash}
\end{itemize}

\subsubsection{Shape definition}
A shape definition class provides two member functions:

\begin{itemize}
\item
\begin{lstlisting}[language=C++, caption={}]
int limit(dim_t axis) const;
\end{lstlisting}
This member function returns for a given axis \( j \) an as small as possible limit \( L_j \) such that:
\[ \forall \boldsymbol{k} \in \mathfrak{K} \,\colon\; k_j \leq L_j \]

\item
\begin{lstlisting}[language=C++, caption={}]
template<class MultiIndex>
int limit(const MultiIndex &base_node, dim_t axis) const;
\end{lstlisting}
This member function for a given axis \( j \) and a given base node \( \boldsymbol{n} \) (whose \( j \)-th entry is zero)
the largest element \( k^\star \) that satisfies: 
\[ \boldsymbol{k} \in \mathfrak{K}, \;
k_i =
   \begin{cases}
      n_i,& i \neq j\\
      k^\star, & i = j
   \end{cases}
\]

\end{itemize}

\subsubsection{Queries}
The default shape enumeration stores all multi-indices in a vector using lexical ordering. The query \emph{ordinal} \(\rightarrow\) \emph{multi-index} is therefore trivial. The query \emph{multi-index} \(\rightarrow\) \emph{ordinal} is done using binary search. Alternatively this implementation can be forced to use \emph{std::unordered\_map} but performance tests revealed that a dictionary is slightly slower than binary seach.

\end{document}

