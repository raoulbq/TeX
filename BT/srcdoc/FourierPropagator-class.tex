%%%%%%%%%%%%%%%%%%%%%%%%%%%%%%%%%%%%%%%%%%%%%%%%%%%%%%%%%%%%%%%%%%%%%%%%%%%
%%                           Class Description                           %%
%%%%%%%%%%%%%%%%%%%%%%%%%%%%%%%%%%%%%%%%%%%%%%%%%%%%%%%%%%%%%%%%%%%%%%%%%%%

    \index{FourierPropagator \textit{(module)}!FourierPropagator \textit{(class)}|(}
\section{Class FourierPropagator}

    \label{FourierPropagator:FourierPropagator}
\begin{tabular}{cccccc}
% Line for Propagator, linespec=[False]
\multicolumn{2}{r}{\settowidth{\BCL}{Propagator}\multirow{2}{\BCL}{Propagator}}
&&
  \\\cline{3-3}
  &&\multicolumn{1}{c|}{}
&&
  \\
&&\multicolumn{2}{l}{\textbf{FourierPropagator}}
\end{tabular}

This class can numerically propagate given initial values
$\Ket{\Psi}$ in a potential surface
$V\ofs{x}$. The propagation is done with a Strang
splitting of the time propagation operator.


%%%%%%%%%%%%%%%%%%%%%%%%%%%%%%%%%%%%%%%%%%%%%%%%%%%%%%%%%%%%%%%%%%%%%%%%%%%
%%                                Methods                                %%
%%%%%%%%%%%%%%%%%%%%%%%%%%%%%%%%%%%%%%%%%%%%%%%%%%%%%%%%%%%%%%%%%%%%%%%%%%%

  \subsection{Methods}

    \vspace{0.5ex}

\hspace{.8\funcindent}\begin{boxedminipage}{\funcwidth}

    \raggedright \textbf{\_\_init\_\_}(\textit{self}, \textit{potential}, \textit{initial\_values})

    \vspace{-1.5ex}

    \rule{\textwidth}{0.5\fboxrule}
\setlength{\parskip}{2ex}
    Initialize a new \textit{FourierPropagator} instance. Precalculate also
    the grid and the propagation operators.

\setlength{\parskip}{1ex}
      \textbf{Parameters}
      \vspace{-1ex}

      \begin{quote}
        \begin{Ventry}{xxxxxxxxxxxxxx}

          \item[potential]

          The potential the state
          $\Ket{\Psi}$ feels during the
          time propagation.

          \item[initial\_values]

          The initial values
          $\Ket{\Psi\ofs{t=0}}$
          given in the canonical basis.

        \end{Ventry}

      \end{quote}

\setlength{\parskip}{1ex}
      \textbf{Raises}
    \vspace{-1ex}

      \begin{quote}
        \begin{description}

          \item[\texttt{ValueError}]

          If the number of components of
          $\Ket{\Psi}$ does not match the
          number of energy levels $\lambda_i$ of the
          potential.

        \end{description}

      \end{quote}

      Overrides: Propagator.\_\_init\_\_

    \end{boxedminipage}

    \vspace{0.5ex}

\hspace{.8\funcindent}\begin{boxedminipage}{\funcwidth}

    \raggedright \textbf{\_\_str\_\_}(\textit{self})

    \vspace{-1.5ex}

    \rule{\textwidth}{0.5\fboxrule}
\setlength{\parskip}{2ex}
    Prepare a printable string representing the \textit{Propagator}
    instance.

\setlength{\parskip}{1ex}

      Overrides: Propagator.\_\_str\_\_

    \end{boxedminipage}

    \vspace{0.5ex}

\hspace{.8\funcindent}\begin{boxedminipage}{\funcwidth}

    \raggedright \textbf{get\_number\_components}(\textit{self})

    \vspace{-1.5ex}

    \rule{\textwidth}{0.5\fboxrule}
\setlength{\parskip}{2ex}
\setlength{\parskip}{1ex}
      \textbf{Return Value}
    \vspace{-1ex}

      \begin{quote}
      The number of components of
      $\Ket{\Psi}$.

      \end{quote}

      Overrides: Propagator.get\_number\_components

    \end{boxedminipage}

    \vspace{0.5ex}

\hspace{.8\funcindent}\begin{boxedminipage}{\funcwidth}

    \raggedright \textbf{get\_potential}(\textit{self})

    \vspace{-1.5ex}

    \rule{\textwidth}{0.5\fboxrule}
\setlength{\parskip}{2ex}
\setlength{\parskip}{1ex}
      \textbf{Return Value}
    \vspace{-1ex}

      \begin{quote}
      The \textit{MatrixPotential} instance used for time propagation.

      \end{quote}

      Overrides: Propagator.get\_potential

    \end{boxedminipage}

    \vspace{0.5ex}

\hspace{.8\funcindent}\begin{boxedminipage}{\funcwidth}

    \raggedright \textbf{get\_wavefunction}(\textit{self})

    \vspace{-1.5ex}

    \rule{\textwidth}{0.5\fboxrule}
\setlength{\parskip}{2ex}
\setlength{\parskip}{1ex}
      \textbf{Return Value}
    \vspace{-1ex}

      \begin{quote}
      The \textit{WaveFunction} instance that stores the current
      wave function data.

      \end{quote}

      Overrides: Propagator.get\_wavefunction

    \end{boxedminipage}

    \label{FourierPropagator:FourierPropagator:get_operators}
    \index{FourierPropagator \textit{(module)}!FourierPropagator \textit{(class)}!FourierPropagator.get\_operators \textit{(method)}}

    \vspace{0.5ex}

\hspace{.8\funcindent}\begin{boxedminipage}{\funcwidth}

    \raggedright \textbf{get\_operators}(\textit{self})

    \vspace{-1.5ex}

    \rule{\textwidth}{0.5\fboxrule}
\setlength{\parskip}{2ex}
\setlength{\parskip}{1ex}
      \textbf{Return Value}
    \vspace{-1ex}

      \begin{quote}
      Return the numerical expressions of the propagation operators $T$
      and $V$.

      \end{quote}

    \end{boxedminipage}

    \vspace{0.5ex}

\hspace{.8\funcindent}\begin{boxedminipage}{\funcwidth}

    \raggedright \textbf{propagate}(\textit{self})

    \vspace{-1.5ex}

    \rule{\textwidth}{0.5\fboxrule}
\setlength{\parskip}{2ex}
    Given the wave function values $\Psi$ at time $t$,
    calculate new values at time $t + \tau$. We perform exactly one
    timestep $\tau$ here.

\setlength{\parskip}{1ex}

      Overrides: Propagator.propagate

    \end{boxedminipage}

    \vspace{0.5ex}

\hspace{.8\funcindent}\begin{boxedminipage}{\funcwidth}

    \raggedright \textbf{kinetic\_energy}(\textit{self}, \textit{summed}={\tt False})

    \vspace{-1.5ex}

    \rule{\textwidth}{0.5\fboxrule}
\setlength{\parskip}{2ex}
    This method just delegates the calculation of kinetic energies to the
    embedded \textit{WaveFunction} object.

\setlength{\parskip}{1ex}
      \textbf{Parameters}
      \vspace{-1ex}

      \begin{quote}
        \begin{Ventry}{xxxxxx}

          \item[summed]

          Whether to sum up the kinetic energies of the individual
          components.

        \end{Ventry}

      \end{quote}

      \textbf{Return Value}
    \vspace{-1ex}

      \begin{quote}
      The kinetic energies.

      \end{quote}

      Overrides: Propagator.kinetic\_energy

    \end{boxedminipage}

    \vspace{0.5ex}

\hspace{.8\funcindent}\begin{boxedminipage}{\funcwidth}

    \raggedright \textbf{potential\_energy}(\textit{self}, \textit{summed}={\tt False})

    \vspace{-1.5ex}

    \rule{\textwidth}{0.5\fboxrule}
\setlength{\parskip}{2ex}
    This method just delegates the calculation of potential energies to the
    embedded \textit{WaveFunction} object.

\setlength{\parskip}{1ex}
      \textbf{Parameters}
      \vspace{-1ex}

      \begin{quote}
        \begin{Ventry}{xxxxxx}

          \item[summed]

          Whether to sum up the potential energies of the individual
          components.

        \end{Ventry}

      \end{quote}

      \textbf{Return Value}
    \vspace{-1ex}

      \begin{quote}
      The potential energies.

      \end{quote}

      Overrides: Propagator.potential\_energy

    \end{boxedminipage}


%%%%%%%%%%%%%%%%%%%%%%%%%%%%%%%%%%%%%%%%%%%%%%%%%%%%%%%%%%%%%%%%%%%%%%%%%%%
%%                          Instance Variables                           %%
%%%%%%%%%%%%%%%%%%%%%%%%%%%%%%%%%%%%%%%%%%%%%%%%%%%%%%%%%%%%%%%%%%%%%%%%%%%

  \subsection{Instance Variables}

    \vspace{-1cm}
\hspace{\varindent}\begin{longtable}{|p{\varnamewidth}|p{\vardescrwidth}|l}
\cline{1-2}
\cline{1-2} \centering \textbf{Name} & \centering \textbf{Description}& \\
\cline{1-2}
\endhead\cline{1-2}\multicolumn{3}{r}{\small\textit{continued on next page}}\\\endfoot\cline{1-2}
\endlastfoot\raggedright p\-o\-t\-e\-n\-t\-i\-a\-l\- & The embedded \textit{MatrixPotential} instance representing the
          potential $V$.&\\
\cline{1-2}
\raggedright P\-s\-i\- & The initial values of the components $\psi_i$
          sampled at the given nodes.&\\
\cline{1-2}
\raggedright n\-o\-d\-e\-s\- & The position space nodes $\gamma$.&\\
\cline{1-2}
\raggedright V\- & The potential operator $V$ defined in position space.&\\
\cline{1-2}
\raggedright o\-m\-e\-g\-a\- & The momentum space nodes $\omega$.&\\
\cline{1-2}
\raggedright T\- & The kinetic operator $T$ defined in momentum space.&\\
\cline{1-2}
\raggedright T\-E\- & Exponential $\exp\ofs{T}$ of
          $T$ used in the Strang splitting.&\\
\cline{1-2}
\raggedright V\-E\- & Exponential $\exp\ofs{V}$ of
          $V$ used in the Strang splitting.&\\
\cline{1-2}
\end{longtable}

    \index{FourierPropagator \textit{(module)}!FourierPropagator \textit{(class)}|)}
